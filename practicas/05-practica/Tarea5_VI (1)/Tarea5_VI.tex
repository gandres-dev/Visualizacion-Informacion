\documentclass[11pt]{article}

    \usepackage[breakable]{tcolorbox}
    \usepackage{parskip} % Stop auto-indenting (to mimic markdown behaviour)
    
    \usepackage{iftex}
    \ifPDFTeX
    	\usepackage[T1]{fontenc}
    	\usepackage{mathpazo}
    \else
    	\usepackage{fontspec}
    \fi

    % Basic figure setup, for now with no caption control since it's done
    % automatically by Pandoc (which extracts ![](path) syntax from Markdown).
    \usepackage{graphicx}
    % Maintain compatibility with old templates. Remove in nbconvert 6.0
    \let\Oldincludegraphics\includegraphics
    % Ensure that by default, figures have no caption (until we provide a
    % proper Figure object with a Caption API and a way to capture that
    % in the conversion process - todo).
    \usepackage{caption}
    \DeclareCaptionFormat{nocaption}{}
    \captionsetup{format=nocaption,aboveskip=0pt,belowskip=0pt}

    \usepackage{float}
    \floatplacement{figure}{H} % forces figures to be placed at the correct location
    \usepackage{xcolor} % Allow colors to be defined
    \usepackage{enumerate} % Needed for markdown enumerations to work
    \usepackage{geometry} % Used to adjust the document margins
    \usepackage{amsmath} % Equations
    \usepackage{amssymb} % Equations
    \usepackage{textcomp} % defines textquotesingle
    % Hack from http://tex.stackexchange.com/a/47451/13684:
    \AtBeginDocument{%
        \def\PYZsq{\textquotesingle}% Upright quotes in Pygmentized code
    }
    \usepackage{upquote} % Upright quotes for verbatim code
    \usepackage{eurosym} % defines \euro
    \usepackage[mathletters]{ucs} % Extended unicode (utf-8) support
    \usepackage{fancyvrb} % verbatim replacement that allows latex
    \usepackage{grffile} % extends the file name processing of package graphics 
                         % to support a larger range
    \makeatletter % fix for old versions of grffile with XeLaTeX
    \@ifpackagelater{grffile}{2019/11/01}
    {
      % Do nothing on new versions
    }
    {
      \def\Gread@@xetex#1{%
        \IfFileExists{"\Gin@base".bb}%
        {\Gread@eps{\Gin@base.bb}}%
        {\Gread@@xetex@aux#1}%
      }
    }
    \makeatother
    \usepackage[Export]{adjustbox} % Used to constrain images to a maximum size
    \adjustboxset{max size={0.9\linewidth}{0.9\paperheight}}

    % The hyperref package gives us a pdf with properly built
    % internal navigation ('pdf bookmarks' for the table of contents,
    % internal cross-reference links, web links for URLs, etc.)
    \usepackage{hyperref}
    % The default LaTeX title has an obnoxious amount of whitespace. By default,
    % titling removes some of it. It also provides customization options.
    \usepackage{titling}
    \usepackage{longtable} % longtable support required by pandoc >1.10
    \usepackage{booktabs}  % table support for pandoc > 1.12.2
    \usepackage[inline]{enumitem} % IRkernel/repr support (it uses the enumerate* environment)
    \usepackage[normalem]{ulem} % ulem is needed to support strikethroughs (\sout)
                                % normalem makes italics be italics, not underlines
    \usepackage{mathrsfs}
    

    
    % Colors for the hyperref package
    \definecolor{urlcolor}{rgb}{0,.145,.698}
    \definecolor{linkcolor}{rgb}{.71,0.21,0.01}
    \definecolor{citecolor}{rgb}{.12,.54,.11}

    % ANSI colors
    \definecolor{ansi-black}{HTML}{3E424D}
    \definecolor{ansi-black-intense}{HTML}{282C36}
    \definecolor{ansi-red}{HTML}{E75C58}
    \definecolor{ansi-red-intense}{HTML}{B22B31}
    \definecolor{ansi-green}{HTML}{00A250}
    \definecolor{ansi-green-intense}{HTML}{007427}
    \definecolor{ansi-yellow}{HTML}{DDB62B}
    \definecolor{ansi-yellow-intense}{HTML}{B27D12}
    \definecolor{ansi-blue}{HTML}{208FFB}
    \definecolor{ansi-blue-intense}{HTML}{0065CA}
    \definecolor{ansi-magenta}{HTML}{D160C4}
    \definecolor{ansi-magenta-intense}{HTML}{A03196}
    \definecolor{ansi-cyan}{HTML}{60C6C8}
    \definecolor{ansi-cyan-intense}{HTML}{258F8F}
    \definecolor{ansi-white}{HTML}{C5C1B4}
    \definecolor{ansi-white-intense}{HTML}{A1A6B2}
    \definecolor{ansi-default-inverse-fg}{HTML}{FFFFFF}
    \definecolor{ansi-default-inverse-bg}{HTML}{000000}

    % common color for the border for error outputs.
    \definecolor{outerrorbackground}{HTML}{FFDFDF}

    % commands and environments needed by pandoc snippets
    % extracted from the output of `pandoc -s`
    \providecommand{\tightlist}{%
      \setlength{\itemsep}{0pt}\setlength{\parskip}{0pt}}
    \DefineVerbatimEnvironment{Highlighting}{Verbatim}{commandchars=\\\{\}}
    % Add ',fontsize=\small' for more characters per line
    \newenvironment{Shaded}{}{}
    \newcommand{\KeywordTok}[1]{\textcolor[rgb]{0.00,0.44,0.13}{\textbf{{#1}}}}
    \newcommand{\DataTypeTok}[1]{\textcolor[rgb]{0.56,0.13,0.00}{{#1}}}
    \newcommand{\DecValTok}[1]{\textcolor[rgb]{0.25,0.63,0.44}{{#1}}}
    \newcommand{\BaseNTok}[1]{\textcolor[rgb]{0.25,0.63,0.44}{{#1}}}
    \newcommand{\FloatTok}[1]{\textcolor[rgb]{0.25,0.63,0.44}{{#1}}}
    \newcommand{\CharTok}[1]{\textcolor[rgb]{0.25,0.44,0.63}{{#1}}}
    \newcommand{\StringTok}[1]{\textcolor[rgb]{0.25,0.44,0.63}{{#1}}}
    \newcommand{\CommentTok}[1]{\textcolor[rgb]{0.38,0.63,0.69}{\textit{{#1}}}}
    \newcommand{\OtherTok}[1]{\textcolor[rgb]{0.00,0.44,0.13}{{#1}}}
    \newcommand{\AlertTok}[1]{\textcolor[rgb]{1.00,0.00,0.00}{\textbf{{#1}}}}
    \newcommand{\FunctionTok}[1]{\textcolor[rgb]{0.02,0.16,0.49}{{#1}}}
    \newcommand{\RegionMarkerTok}[1]{{#1}}
    \newcommand{\ErrorTok}[1]{\textcolor[rgb]{1.00,0.00,0.00}{\textbf{{#1}}}}
    \newcommand{\NormalTok}[1]{{#1}}
    
    % Additional commands for more recent versions of Pandoc
    \newcommand{\ConstantTok}[1]{\textcolor[rgb]{0.53,0.00,0.00}{{#1}}}
    \newcommand{\SpecialCharTok}[1]{\textcolor[rgb]{0.25,0.44,0.63}{{#1}}}
    \newcommand{\VerbatimStringTok}[1]{\textcolor[rgb]{0.25,0.44,0.63}{{#1}}}
    \newcommand{\SpecialStringTok}[1]{\textcolor[rgb]{0.73,0.40,0.53}{{#1}}}
    \newcommand{\ImportTok}[1]{{#1}}
    \newcommand{\DocumentationTok}[1]{\textcolor[rgb]{0.73,0.13,0.13}{\textit{{#1}}}}
    \newcommand{\AnnotationTok}[1]{\textcolor[rgb]{0.38,0.63,0.69}{\textbf{\textit{{#1}}}}}
    \newcommand{\CommentVarTok}[1]{\textcolor[rgb]{0.38,0.63,0.69}{\textbf{\textit{{#1}}}}}
    \newcommand{\VariableTok}[1]{\textcolor[rgb]{0.10,0.09,0.49}{{#1}}}
    \newcommand{\ControlFlowTok}[1]{\textcolor[rgb]{0.00,0.44,0.13}{\textbf{{#1}}}}
    \newcommand{\OperatorTok}[1]{\textcolor[rgb]{0.40,0.40,0.40}{{#1}}}
    \newcommand{\BuiltInTok}[1]{{#1}}
    \newcommand{\ExtensionTok}[1]{{#1}}
    \newcommand{\PreprocessorTok}[1]{\textcolor[rgb]{0.74,0.48,0.00}{{#1}}}
    \newcommand{\AttributeTok}[1]{\textcolor[rgb]{0.49,0.56,0.16}{{#1}}}
    \newcommand{\InformationTok}[1]{\textcolor[rgb]{0.38,0.63,0.69}{\textbf{\textit{{#1}}}}}
    \newcommand{\WarningTok}[1]{\textcolor[rgb]{0.38,0.63,0.69}{\textbf{\textit{{#1}}}}}
    
    
    % Define a nice break command that doesn't care if a line doesn't already
    % exist.
    \def\br{\hspace*{\fill} \\* }
    % Math Jax compatibility definitions
    \def\gt{>}
    \def\lt{<}
    \let\Oldtex\TeX
    \let\Oldlatex\LaTeX
    \renewcommand{\TeX}{\textrm{\Oldtex}}
    \renewcommand{\LaTeX}{\textrm{\Oldlatex}}
    % Document parameters
    % Document title
    \title{Tarea5\_VI}
    
    
    
    
    
% Pygments definitions
\makeatletter
\def\PY@reset{\let\PY@it=\relax \let\PY@bf=\relax%
    \let\PY@ul=\relax \let\PY@tc=\relax%
    \let\PY@bc=\relax \let\PY@ff=\relax}
\def\PY@tok#1{\csname PY@tok@#1\endcsname}
\def\PY@toks#1+{\ifx\relax#1\empty\else%
    \PY@tok{#1}\expandafter\PY@toks\fi}
\def\PY@do#1{\PY@bc{\PY@tc{\PY@ul{%
    \PY@it{\PY@bf{\PY@ff{#1}}}}}}}
\def\PY#1#2{\PY@reset\PY@toks#1+\relax+\PY@do{#2}}

\@namedef{PY@tok@w}{\def\PY@tc##1{\textcolor[rgb]{0.73,0.73,0.73}{##1}}}
\@namedef{PY@tok@c}{\let\PY@it=\textit\def\PY@tc##1{\textcolor[rgb]{0.25,0.50,0.50}{##1}}}
\@namedef{PY@tok@cp}{\def\PY@tc##1{\textcolor[rgb]{0.74,0.48,0.00}{##1}}}
\@namedef{PY@tok@k}{\let\PY@bf=\textbf\def\PY@tc##1{\textcolor[rgb]{0.00,0.50,0.00}{##1}}}
\@namedef{PY@tok@kp}{\def\PY@tc##1{\textcolor[rgb]{0.00,0.50,0.00}{##1}}}
\@namedef{PY@tok@kt}{\def\PY@tc##1{\textcolor[rgb]{0.69,0.00,0.25}{##1}}}
\@namedef{PY@tok@o}{\def\PY@tc##1{\textcolor[rgb]{0.40,0.40,0.40}{##1}}}
\@namedef{PY@tok@ow}{\let\PY@bf=\textbf\def\PY@tc##1{\textcolor[rgb]{0.67,0.13,1.00}{##1}}}
\@namedef{PY@tok@nb}{\def\PY@tc##1{\textcolor[rgb]{0.00,0.50,0.00}{##1}}}
\@namedef{PY@tok@nf}{\def\PY@tc##1{\textcolor[rgb]{0.00,0.00,1.00}{##1}}}
\@namedef{PY@tok@nc}{\let\PY@bf=\textbf\def\PY@tc##1{\textcolor[rgb]{0.00,0.00,1.00}{##1}}}
\@namedef{PY@tok@nn}{\let\PY@bf=\textbf\def\PY@tc##1{\textcolor[rgb]{0.00,0.00,1.00}{##1}}}
\@namedef{PY@tok@ne}{\let\PY@bf=\textbf\def\PY@tc##1{\textcolor[rgb]{0.82,0.25,0.23}{##1}}}
\@namedef{PY@tok@nv}{\def\PY@tc##1{\textcolor[rgb]{0.10,0.09,0.49}{##1}}}
\@namedef{PY@tok@no}{\def\PY@tc##1{\textcolor[rgb]{0.53,0.00,0.00}{##1}}}
\@namedef{PY@tok@nl}{\def\PY@tc##1{\textcolor[rgb]{0.63,0.63,0.00}{##1}}}
\@namedef{PY@tok@ni}{\let\PY@bf=\textbf\def\PY@tc##1{\textcolor[rgb]{0.60,0.60,0.60}{##1}}}
\@namedef{PY@tok@na}{\def\PY@tc##1{\textcolor[rgb]{0.49,0.56,0.16}{##1}}}
\@namedef{PY@tok@nt}{\let\PY@bf=\textbf\def\PY@tc##1{\textcolor[rgb]{0.00,0.50,0.00}{##1}}}
\@namedef{PY@tok@nd}{\def\PY@tc##1{\textcolor[rgb]{0.67,0.13,1.00}{##1}}}
\@namedef{PY@tok@s}{\def\PY@tc##1{\textcolor[rgb]{0.73,0.13,0.13}{##1}}}
\@namedef{PY@tok@sd}{\let\PY@it=\textit\def\PY@tc##1{\textcolor[rgb]{0.73,0.13,0.13}{##1}}}
\@namedef{PY@tok@si}{\let\PY@bf=\textbf\def\PY@tc##1{\textcolor[rgb]{0.73,0.40,0.53}{##1}}}
\@namedef{PY@tok@se}{\let\PY@bf=\textbf\def\PY@tc##1{\textcolor[rgb]{0.73,0.40,0.13}{##1}}}
\@namedef{PY@tok@sr}{\def\PY@tc##1{\textcolor[rgb]{0.73,0.40,0.53}{##1}}}
\@namedef{PY@tok@ss}{\def\PY@tc##1{\textcolor[rgb]{0.10,0.09,0.49}{##1}}}
\@namedef{PY@tok@sx}{\def\PY@tc##1{\textcolor[rgb]{0.00,0.50,0.00}{##1}}}
\@namedef{PY@tok@m}{\def\PY@tc##1{\textcolor[rgb]{0.40,0.40,0.40}{##1}}}
\@namedef{PY@tok@gh}{\let\PY@bf=\textbf\def\PY@tc##1{\textcolor[rgb]{0.00,0.00,0.50}{##1}}}
\@namedef{PY@tok@gu}{\let\PY@bf=\textbf\def\PY@tc##1{\textcolor[rgb]{0.50,0.00,0.50}{##1}}}
\@namedef{PY@tok@gd}{\def\PY@tc##1{\textcolor[rgb]{0.63,0.00,0.00}{##1}}}
\@namedef{PY@tok@gi}{\def\PY@tc##1{\textcolor[rgb]{0.00,0.63,0.00}{##1}}}
\@namedef{PY@tok@gr}{\def\PY@tc##1{\textcolor[rgb]{1.00,0.00,0.00}{##1}}}
\@namedef{PY@tok@ge}{\let\PY@it=\textit}
\@namedef{PY@tok@gs}{\let\PY@bf=\textbf}
\@namedef{PY@tok@gp}{\let\PY@bf=\textbf\def\PY@tc##1{\textcolor[rgb]{0.00,0.00,0.50}{##1}}}
\@namedef{PY@tok@go}{\def\PY@tc##1{\textcolor[rgb]{0.53,0.53,0.53}{##1}}}
\@namedef{PY@tok@gt}{\def\PY@tc##1{\textcolor[rgb]{0.00,0.27,0.87}{##1}}}
\@namedef{PY@tok@err}{\def\PY@bc##1{{\setlength{\fboxsep}{\string -\fboxrule}\fcolorbox[rgb]{1.00,0.00,0.00}{1,1,1}{\strut ##1}}}}
\@namedef{PY@tok@kc}{\let\PY@bf=\textbf\def\PY@tc##1{\textcolor[rgb]{0.00,0.50,0.00}{##1}}}
\@namedef{PY@tok@kd}{\let\PY@bf=\textbf\def\PY@tc##1{\textcolor[rgb]{0.00,0.50,0.00}{##1}}}
\@namedef{PY@tok@kn}{\let\PY@bf=\textbf\def\PY@tc##1{\textcolor[rgb]{0.00,0.50,0.00}{##1}}}
\@namedef{PY@tok@kr}{\let\PY@bf=\textbf\def\PY@tc##1{\textcolor[rgb]{0.00,0.50,0.00}{##1}}}
\@namedef{PY@tok@bp}{\def\PY@tc##1{\textcolor[rgb]{0.00,0.50,0.00}{##1}}}
\@namedef{PY@tok@fm}{\def\PY@tc##1{\textcolor[rgb]{0.00,0.00,1.00}{##1}}}
\@namedef{PY@tok@vc}{\def\PY@tc##1{\textcolor[rgb]{0.10,0.09,0.49}{##1}}}
\@namedef{PY@tok@vg}{\def\PY@tc##1{\textcolor[rgb]{0.10,0.09,0.49}{##1}}}
\@namedef{PY@tok@vi}{\def\PY@tc##1{\textcolor[rgb]{0.10,0.09,0.49}{##1}}}
\@namedef{PY@tok@vm}{\def\PY@tc##1{\textcolor[rgb]{0.10,0.09,0.49}{##1}}}
\@namedef{PY@tok@sa}{\def\PY@tc##1{\textcolor[rgb]{0.73,0.13,0.13}{##1}}}
\@namedef{PY@tok@sb}{\def\PY@tc##1{\textcolor[rgb]{0.73,0.13,0.13}{##1}}}
\@namedef{PY@tok@sc}{\def\PY@tc##1{\textcolor[rgb]{0.73,0.13,0.13}{##1}}}
\@namedef{PY@tok@dl}{\def\PY@tc##1{\textcolor[rgb]{0.73,0.13,0.13}{##1}}}
\@namedef{PY@tok@s2}{\def\PY@tc##1{\textcolor[rgb]{0.73,0.13,0.13}{##1}}}
\@namedef{PY@tok@sh}{\def\PY@tc##1{\textcolor[rgb]{0.73,0.13,0.13}{##1}}}
\@namedef{PY@tok@s1}{\def\PY@tc##1{\textcolor[rgb]{0.73,0.13,0.13}{##1}}}
\@namedef{PY@tok@mb}{\def\PY@tc##1{\textcolor[rgb]{0.40,0.40,0.40}{##1}}}
\@namedef{PY@tok@mf}{\def\PY@tc##1{\textcolor[rgb]{0.40,0.40,0.40}{##1}}}
\@namedef{PY@tok@mh}{\def\PY@tc##1{\textcolor[rgb]{0.40,0.40,0.40}{##1}}}
\@namedef{PY@tok@mi}{\def\PY@tc##1{\textcolor[rgb]{0.40,0.40,0.40}{##1}}}
\@namedef{PY@tok@il}{\def\PY@tc##1{\textcolor[rgb]{0.40,0.40,0.40}{##1}}}
\@namedef{PY@tok@mo}{\def\PY@tc##1{\textcolor[rgb]{0.40,0.40,0.40}{##1}}}
\@namedef{PY@tok@ch}{\let\PY@it=\textit\def\PY@tc##1{\textcolor[rgb]{0.25,0.50,0.50}{##1}}}
\@namedef{PY@tok@cm}{\let\PY@it=\textit\def\PY@tc##1{\textcolor[rgb]{0.25,0.50,0.50}{##1}}}
\@namedef{PY@tok@cpf}{\let\PY@it=\textit\def\PY@tc##1{\textcolor[rgb]{0.25,0.50,0.50}{##1}}}
\@namedef{PY@tok@c1}{\let\PY@it=\textit\def\PY@tc##1{\textcolor[rgb]{0.25,0.50,0.50}{##1}}}
\@namedef{PY@tok@cs}{\let\PY@it=\textit\def\PY@tc##1{\textcolor[rgb]{0.25,0.50,0.50}{##1}}}

\def\PYZbs{\char`\\}
\def\PYZus{\char`\_}
\def\PYZob{\char`\{}
\def\PYZcb{\char`\}}
\def\PYZca{\char`\^}
\def\PYZam{\char`\&}
\def\PYZlt{\char`\<}
\def\PYZgt{\char`\>}
\def\PYZsh{\char`\#}
\def\PYZpc{\char`\%}
\def\PYZdl{\char`\$}
\def\PYZhy{\char`\-}
\def\PYZsq{\char`\'}
\def\PYZdq{\char`\"}
\def\PYZti{\char`\~}
% for compatibility with earlier versions
\def\PYZat{@}
\def\PYZlb{[}
\def\PYZrb{]}
\makeatother


    % For linebreaks inside Verbatim environment from package fancyvrb. 
    \makeatletter
        \newbox\Wrappedcontinuationbox 
        \newbox\Wrappedvisiblespacebox 
        \newcommand*\Wrappedvisiblespace {\textcolor{red}{\textvisiblespace}} 
        \newcommand*\Wrappedcontinuationsymbol {\textcolor{red}{\llap{\tiny$\m@th\hookrightarrow$}}} 
        \newcommand*\Wrappedcontinuationindent {3ex } 
        \newcommand*\Wrappedafterbreak {\kern\Wrappedcontinuationindent\copy\Wrappedcontinuationbox} 
        % Take advantage of the already applied Pygments mark-up to insert 
        % potential linebreaks for TeX processing. 
        %        {, <, #, %, $, ' and ": go to next line. 
        %        _, }, ^, &, >, - and ~: stay at end of broken line. 
        % Use of \textquotesingle for straight quote. 
        \newcommand*\Wrappedbreaksatspecials {% 
            \def\PYGZus{\discretionary{\char`\_}{\Wrappedafterbreak}{\char`\_}}% 
            \def\PYGZob{\discretionary{}{\Wrappedafterbreak\char`\{}{\char`\{}}% 
            \def\PYGZcb{\discretionary{\char`\}}{\Wrappedafterbreak}{\char`\}}}% 
            \def\PYGZca{\discretionary{\char`\^}{\Wrappedafterbreak}{\char`\^}}% 
            \def\PYGZam{\discretionary{\char`\&}{\Wrappedafterbreak}{\char`\&}}% 
            \def\PYGZlt{\discretionary{}{\Wrappedafterbreak\char`\<}{\char`\<}}% 
            \def\PYGZgt{\discretionary{\char`\>}{\Wrappedafterbreak}{\char`\>}}% 
            \def\PYGZsh{\discretionary{}{\Wrappedafterbreak\char`\#}{\char`\#}}% 
            \def\PYGZpc{\discretionary{}{\Wrappedafterbreak\char`\%}{\char`\%}}% 
            \def\PYGZdl{\discretionary{}{\Wrappedafterbreak\char`\$}{\char`\$}}% 
            \def\PYGZhy{\discretionary{\char`\-}{\Wrappedafterbreak}{\char`\-}}% 
            \def\PYGZsq{\discretionary{}{\Wrappedafterbreak\textquotesingle}{\textquotesingle}}% 
            \def\PYGZdq{\discretionary{}{\Wrappedafterbreak\char`\"}{\char`\"}}% 
            \def\PYGZti{\discretionary{\char`\~}{\Wrappedafterbreak}{\char`\~}}% 
        } 
        % Some characters . , ; ? ! / are not pygmentized. 
        % This macro makes them "active" and they will insert potential linebreaks 
        \newcommand*\Wrappedbreaksatpunct {% 
            \lccode`\~`\.\lowercase{\def~}{\discretionary{\hbox{\char`\.}}{\Wrappedafterbreak}{\hbox{\char`\.}}}% 
            \lccode`\~`\,\lowercase{\def~}{\discretionary{\hbox{\char`\,}}{\Wrappedafterbreak}{\hbox{\char`\,}}}% 
            \lccode`\~`\;\lowercase{\def~}{\discretionary{\hbox{\char`\;}}{\Wrappedafterbreak}{\hbox{\char`\;}}}% 
            \lccode`\~`\:\lowercase{\def~}{\discretionary{\hbox{\char`\:}}{\Wrappedafterbreak}{\hbox{\char`\:}}}% 
            \lccode`\~`\?\lowercase{\def~}{\discretionary{\hbox{\char`\?}}{\Wrappedafterbreak}{\hbox{\char`\?}}}% 
            \lccode`\~`\!\lowercase{\def~}{\discretionary{\hbox{\char`\!}}{\Wrappedafterbreak}{\hbox{\char`\!}}}% 
            \lccode`\~`\/\lowercase{\def~}{\discretionary{\hbox{\char`\/}}{\Wrappedafterbreak}{\hbox{\char`\/}}}% 
            \catcode`\.\active
            \catcode`\,\active 
            \catcode`\;\active
            \catcode`\:\active
            \catcode`\?\active
            \catcode`\!\active
            \catcode`\/\active 
            \lccode`\~`\~ 	
        }
    \makeatother

    \let\OriginalVerbatim=\Verbatim
    \makeatletter
    \renewcommand{\Verbatim}[1][1]{%
        %\parskip\z@skip
        \sbox\Wrappedcontinuationbox {\Wrappedcontinuationsymbol}%
        \sbox\Wrappedvisiblespacebox {\FV@SetupFont\Wrappedvisiblespace}%
        \def\FancyVerbFormatLine ##1{\hsize\linewidth
            \vtop{\raggedright\hyphenpenalty\z@\exhyphenpenalty\z@
                \doublehyphendemerits\z@\finalhyphendemerits\z@
                \strut ##1\strut}%
        }%
        % If the linebreak is at a space, the latter will be displayed as visible
        % space at end of first line, and a continuation symbol starts next line.
        % Stretch/shrink are however usually zero for typewriter font.
        \def\FV@Space {%
            \nobreak\hskip\z@ plus\fontdimen3\font minus\fontdimen4\font
            \discretionary{\copy\Wrappedvisiblespacebox}{\Wrappedafterbreak}
            {\kern\fontdimen2\font}%
        }%
        
        % Allow breaks at special characters using \PYG... macros.
        \Wrappedbreaksatspecials
        % Breaks at punctuation characters . , ; ? ! and / need catcode=\active 	
        \OriginalVerbatim[#1,codes*=\Wrappedbreaksatpunct]%
    }
    \makeatother

    % Exact colors from NB
    \definecolor{incolor}{HTML}{303F9F}
    \definecolor{outcolor}{HTML}{D84315}
    \definecolor{cellborder}{HTML}{CFCFCF}
    \definecolor{cellbackground}{HTML}{F7F7F7}
    
    % prompt
    \makeatletter
    \newcommand{\boxspacing}{\kern\kvtcb@left@rule\kern\kvtcb@boxsep}
    \makeatother
    \newcommand{\prompt}[4]{
        {\ttfamily\llap{{\color{#2}[#3]:\hspace{3pt}#4}}\vspace{-\baselineskip}}
    }
    

    
    % Prevent overflowing lines due to hard-to-break entities
    \sloppy 
    % Setup hyperref package
    \hypersetup{
      breaklinks=true,  % so long urls are correctly broken across lines
      colorlinks=true,
      urlcolor=urlcolor,
      linkcolor=linkcolor,
      citecolor=citecolor,
      }
    % Slightly bigger margins than the latex defaults
    
    \geometry{verbose,tmargin=1in,bmargin=1in,lmargin=1in,rmargin=1in}
    
    

\begin{document}
    
    \maketitle
    
    

    
    \begin{tcolorbox}[breakable, size=fbox, boxrule=1pt, pad at break*=1mm,colback=cellbackground, colframe=cellborder]
\prompt{In}{incolor}{389}{\boxspacing}
\begin{Verbatim}[commandchars=\\\{\}]
\PY{k+kn}{import} \PY{n+nn}{numpy} \PY{k}{as} \PY{n+nn}{np}
\PY{k+kn}{import} \PY{n+nn}{pandas} \PY{k}{as} \PY{n+nn}{pd}
\PY{k+kn}{import} \PY{n+nn}{matplotlib}\PY{n+nn}{.}\PY{n+nn}{pyplot} \PY{k}{as} \PY{n+nn}{plt}
\PY{k+kn}{import} \PY{n+nn}{seaborn} \PY{k}{as} \PY{n+nn}{sns}
\end{Verbatim}
\end{tcolorbox}

    \hypertarget{visualizaciuxf3n-de-la-informaciuxf3n---tarea-5}{%
\section{Visualización de la Información - Tarea
5}\label{visualizaciuxf3n-de-la-informaciuxf3n---tarea-5}}

\textbf{Andrés Urbano Guillermo Gerardo}

El hundimiento del Titanic es uno de los naufragios más infames de la
historia. Esta es la descripción de las variables de nuestro conjunto de
datos.

\begin{longtable}[]{@{}ll@{}}
\toprule
Caracteristica & Descripcion \\
\midrule
\endhead
pclass & Clase del pasajero \\
survived & Indica si sobrevivio (0 = No; 1 = Sí) \\
name & Nombre \\
sex & Una variable categorica de hombre y mujer \\
age & edad en años \\
sibsp & Número de hermanos/cónyuges a bordo \\
parch & Número de padres/niños a bordo \\
ticket & Numero de ticket \\
fare & Tarifa de pasajero \\
cabin & Cabina \\
embarked & El lugar donde embarcaron (Cherbourg, Queenstowny
Southampton) \\
boat & Bote salvavidas \\
body & Número de identificación del cuerpo \\
home.dest & Inicio/Destino \\
\bottomrule
\end{longtable}

    \begin{tcolorbox}[breakable, size=fbox, boxrule=1pt, pad at break*=1mm,colback=cellbackground, colframe=cellborder]
\prompt{In}{incolor}{390}{\boxspacing}
\begin{Verbatim}[commandchars=\\\{\}]
\PY{n}{data} \PY{o}{=} \PY{n}{pd}\PY{o}{.}\PY{n}{read\PYZus{}csv}\PY{p}{(}\PY{l+s+s1}{\PYZsq{}}\PY{l+s+s1}{data/titanic3.csv}\PY{l+s+s1}{\PYZsq{}}\PY{p}{)}
\PY{n}{data}\PY{o}{.}\PY{n}{tail}\PY{p}{(}\PY{p}{)}
\end{Verbatim}
\end{tcolorbox}

            \begin{tcolorbox}[breakable, size=fbox, boxrule=.5pt, pad at break*=1mm, opacityfill=0]
\prompt{Out}{outcolor}{390}{\boxspacing}
\begin{Verbatim}[commandchars=\\\{\}]
      pclass  survived                       name     sex   age  sibsp  parch  \textbackslash{}
1304       3         0       Zabour, Miss. Hileni  female  14.5      1      0
1305       3         0      Zabour, Miss. Thamine  female   NaN      1      0
1306       3         0  Zakarian, Mr. Mapriededer    male  26.5      0      0
1307       3         0        Zakarian, Mr. Ortin    male  27.0      0      0
1308       3         0         Zimmerman, Mr. Leo    male  29.0      0      0

      ticket     fare cabin embarked boat   body home.dest
1304    2665  14.4542   NaN        C  NaN  328.0       NaN
1305    2665  14.4542   NaN        C  NaN    NaN       NaN
1306    2656   7.2250   NaN        C  NaN  304.0       NaN
1307    2670   7.2250   NaN        C  NaN    NaN       NaN
1308  315082   7.8750   NaN        S  NaN    NaN       NaN
\end{Verbatim}
\end{tcolorbox}
        
    \begin{enumerate}
\def\labelenumi{\roman{enumi})}
\tightlist
\item
  ¿Cuántos cuerpos fueron encontrados?
\end{enumerate}

Para conocer la cantidad de cuerpos encontrados dado nuestro conjunto
datos debemos de filtrar la columna body.

    \begin{tcolorbox}[breakable, size=fbox, boxrule=1pt, pad at break*=1mm,colback=cellbackground, colframe=cellborder]
\prompt{In}{incolor}{391}{\boxspacing}
\begin{Verbatim}[commandchars=\\\{\}]
\PY{n}{body\PYZus{}found} \PY{o}{=} \PY{n}{data}\PY{o}{.}\PY{n}{body}\PY{o}{.}\PY{n}{dropna}\PY{p}{(}\PY{p}{)}
\PY{n+nb}{print}\PY{p}{(}\PY{l+s+sa}{f}\PY{l+s+s2}{\PYZdq{}}\PY{l+s+s2}{Cantidad de cuerpos encontrados: }\PY{l+s+si}{\PYZob{}}\PY{n+nb}{len}\PY{p}{(}\PY{n}{body\PYZus{}found}\PY{p}{)}\PY{l+s+si}{\PYZcb{}}\PY{l+s+s2}{\PYZdq{}}\PY{p}{)}
\end{Verbatim}
\end{tcolorbox}

    \begin{Verbatim}[commandchars=\\\{\}]
Cantidad de cuerpos encontrados: 121
    \end{Verbatim}

    Tambien podriamos conocer la cantidad de personas que sobrevieron y que
en clase pertenecian:

    \begin{tcolorbox}[breakable, size=fbox, boxrule=1pt, pad at break*=1mm,colback=cellbackground, colframe=cellborder]
\prompt{In}{incolor}{392}{\boxspacing}
\begin{Verbatim}[commandchars=\\\{\}]
\PY{n}{survived\PYZus{}people} \PY{o}{=} \PY{n}{data}\PY{p}{[}\PY{n}{data}\PY{o}{.}\PY{n}{survived} \PY{o}{==} \PY{l+m+mi}{1}\PY{p}{]}
\PY{n}{no\PYZus{}survived\PYZus{}people} \PY{o}{=} \PY{n}{data}\PY{p}{[}\PY{n}{data}\PY{o}{.}\PY{n}{survived} \PY{o}{==} \PY{l+m+mi}{0}\PY{p}{]}
\PY{n+nb}{print}\PY{p}{(}\PY{l+s+sa}{f}\PY{l+s+s2}{\PYZdq{}}\PY{l+s+s2}{Total de personas encontradas }\PY{l+s+si}{\PYZob{}}\PY{n+nb}{len}\PY{p}{(}\PY{n}{survived\PYZus{}people}\PY{p}{)}\PY{l+s+si}{\PYZcb{}}\PY{l+s+s2}{\PYZdq{}}\PY{p}{)}
\end{Verbatim}
\end{tcolorbox}

    \begin{Verbatim}[commandchars=\\\{\}]
Total de personas encontradas 500
    \end{Verbatim}

    \begin{tcolorbox}[breakable, size=fbox, boxrule=1pt, pad at break*=1mm,colback=cellbackground, colframe=cellborder]
\prompt{In}{incolor}{393}{\boxspacing}
\begin{Verbatim}[commandchars=\\\{\}]
\PY{n}{sns}\PY{o}{.}\PY{n}{catplot}\PY{p}{(}\PY{n}{data}\PY{o}{=}\PY{n}{data}\PY{p}{,} \PY{n}{kind}\PY{o}{=}\PY{l+s+s2}{\PYZdq{}}\PY{l+s+s2}{bar}\PY{l+s+s2}{\PYZdq{}}\PY{p}{,} \PY{n}{x}\PY{o}{=}\PY{l+s+s2}{\PYZdq{}}\PY{l+s+s2}{sex}\PY{l+s+s2}{\PYZdq{}}\PY{p}{,} \PY{n}{y}\PY{o}{=}\PY{l+s+s2}{\PYZdq{}}\PY{l+s+s2}{survived}\PY{l+s+s2}{\PYZdq{}}\PY{p}{,} \PY{n}{hue}\PY{o}{=}\PY{l+s+s2}{\PYZdq{}}\PY{l+s+s2}{pclass}\PY{l+s+s2}{\PYZdq{}}\PY{p}{)}
\end{Verbatim}
\end{tcolorbox}

            \begin{tcolorbox}[breakable, size=fbox, boxrule=.5pt, pad at break*=1mm, opacityfill=0]
\prompt{Out}{outcolor}{393}{\boxspacing}
\begin{Verbatim}[commandchars=\\\{\}]
<seaborn.axisgrid.FacetGrid at 0x7f5a8fe9f220>
\end{Verbatim}
\end{tcolorbox}
        
    \begin{center}
    \adjustimage{max size={0.9\linewidth}{0.9\paperheight}}{output_7_1.png}
    \end{center}
    { \hspace*{\fill} \\}
    
    \begin{enumerate}
\def\labelenumi{\roman{enumi})}
\setcounter{enumi}{1}
\tightlist
\item
  ¿Cuántos de ellos fueron hombres mayores a cuarenta años?
\end{enumerate}

    \begin{tcolorbox}[breakable, size=fbox, boxrule=1pt, pad at break*=1mm,colback=cellbackground, colframe=cellborder]
\prompt{In}{incolor}{394}{\boxspacing}
\begin{Verbatim}[commandchars=\\\{\}]
\PY{c+c1}{\PYZsh{}  Filtramos por edad y sexo}
\PY{n}{male\PYZus{}40} \PY{o}{=} \PY{n}{data}\PY{p}{[}\PY{p}{(}\PY{n}{data}\PY{o}{.}\PY{n}{age} \PY{o}{\PYZgt{}} \PY{l+m+mi}{40}\PY{p}{)} \PY{o}{\PYZam{}} \PY{p}{(}\PY{n}{data}\PY{o}{.}\PY{n}{sex} \PY{o}{==} \PY{l+s+s2}{\PYZdq{}}\PY{l+s+s2}{male}\PY{l+s+s2}{\PYZdq{}}\PY{p}{)}\PY{p}{]}
\PY{n}{sns}\PY{o}{.}\PY{n}{displot}\PY{p}{(}\PY{n}{data}\PY{o}{=}\PY{n}{male\PYZus{}40}\PY{p}{,} \PY{n}{x}\PY{o}{=}\PY{l+s+s2}{\PYZdq{}}\PY{l+s+s2}{age}\PY{l+s+s2}{\PYZdq{}}\PY{p}{)}
\end{Verbatim}
\end{tcolorbox}

            \begin{tcolorbox}[breakable, size=fbox, boxrule=.5pt, pad at break*=1mm, opacityfill=0]
\prompt{Out}{outcolor}{394}{\boxspacing}
\begin{Verbatim}[commandchars=\\\{\}]
<seaborn.axisgrid.FacetGrid at 0x7f5a90a020b0>
\end{Verbatim}
\end{tcolorbox}
        
    \begin{center}
    \adjustimage{max size={0.9\linewidth}{0.9\paperheight}}{output_9_1.png}
    \end{center}
    { \hspace*{\fill} \\}
    
    \begin{enumerate}
\def\labelenumi{\roman{enumi})}
\setcounter{enumi}{2}
\tightlist
\item
  ¿Cuantas mujeres desaparecieron entre las edades de 15 a 35 años? Para
  deternmiar las mujeres que desaparecieron es encontrar la cantidad de
  personas que no sobrevivieron en el accidente:
\end{enumerate}

    \begin{tcolorbox}[breakable, size=fbox, boxrule=1pt, pad at break*=1mm,colback=cellbackground, colframe=cellborder]
\prompt{In}{incolor}{395}{\boxspacing}
\begin{Verbatim}[commandchars=\\\{\}]
\PY{n}{missing\PYZus{}woman} \PY{o}{=} \PY{n}{data}\PY{p}{[}\PY{p}{(}\PY{n}{data}\PY{o}{.}\PY{n}{sex} \PY{o}{==} \PY{l+s+s2}{\PYZdq{}}\PY{l+s+s2}{female}\PY{l+s+s2}{\PYZdq{}}\PY{p}{)} \PY{o}{\PYZam{}} \PY{p}{(}\PY{n}{data}\PY{o}{.}\PY{n}{age} \PY{o}{\PYZgt{}}\PY{o}{=} \PY{l+m+mi}{15}\PY{p}{)} \PY{o}{\PYZam{}} \PY{p}{(}\PY{n}{data}\PY{o}{.}\PY{n}{age} \PY{o}{\PYZlt{}}\PY{o}{=} \PY{l+m+mi}{35}\PY{p}{)} \PY{o}{\PYZam{}} \PY{p}{(}\PY{n}{data}\PY{o}{.}\PY{n}{survived} \PY{o}{==} \PY{l+m+mi}{0}\PY{p}{)}\PY{p}{]}
\PY{n+nb}{print}\PY{p}{(}\PY{l+s+sa}{f}\PY{l+s+s2}{\PYZdq{}}\PY{l+s+s2}{Total de mujeres desaparecidas }\PY{l+s+si}{\PYZob{}}\PY{n+nb}{len}\PY{p}{(}\PY{n}{missing\PYZus{}woman}\PY{p}{)}\PY{l+s+si}{\PYZcb{}}\PY{l+s+s2}{\PYZdq{}}\PY{p}{)}
\end{Verbatim}
\end{tcolorbox}

    \begin{Verbatim}[commandchars=\\\{\}]
Total de mujeres desaparecidas 55
    \end{Verbatim}

    \begin{tcolorbox}[breakable, size=fbox, boxrule=1pt, pad at break*=1mm,colback=cellbackground, colframe=cellborder]
\prompt{In}{incolor}{396}{\boxspacing}
\begin{Verbatim}[commandchars=\\\{\}]
\PY{n}{sns}\PY{o}{.}\PY{n}{catplot}\PY{p}{(}\PY{n}{x}\PY{o}{=}\PY{l+s+s2}{\PYZdq{}}\PY{l+s+s2}{pclass}\PY{l+s+s2}{\PYZdq{}}\PY{p}{,} \PY{n}{y}\PY{o}{=}\PY{l+s+s2}{\PYZdq{}}\PY{l+s+s2}{age}\PY{l+s+s2}{\PYZdq{}}\PY{p}{,} \PY{n}{data}\PY{o}{=}\PY{n}{missing\PYZus{}woman}\PY{p}{)}
\end{Verbatim}
\end{tcolorbox}

            \begin{tcolorbox}[breakable, size=fbox, boxrule=.5pt, pad at break*=1mm, opacityfill=0]
\prompt{Out}{outcolor}{396}{\boxspacing}
\begin{Verbatim}[commandchars=\\\{\}]
<seaborn.axisgrid.FacetGrid at 0x7f5a90120c40>
\end{Verbatim}
\end{tcolorbox}
        
    \begin{center}
    \adjustimage{max size={0.9\linewidth}{0.9\paperheight}}{output_12_1.png}
    \end{center}
    { \hspace*{\fill} \\}
    
    Por otra parte, tambien podemos conocer las mujeres que no encontraron
sus cuerpos.

    \begin{tcolorbox}[breakable, size=fbox, boxrule=1pt, pad at break*=1mm,colback=cellbackground, colframe=cellborder]
\prompt{In}{incolor}{397}{\boxspacing}
\begin{Verbatim}[commandchars=\\\{\}]
\PY{n}{not\PYZus{}found\PYZus{}woman} \PY{o}{=} \PY{n}{missing\PYZus{}woman}\PY{o}{.}\PY{n}{body}\PY{o}{.}\PY{n}{fillna}\PY{p}{(}\PY{l+s+s2}{\PYZdq{}}\PY{l+s+s2}{Cuerpos no encontrados}\PY{l+s+s2}{\PYZdq{}}\PY{p}{)}
\PY{n}{not\PYZus{}found\PYZus{}woman} \PY{o}{=} \PY{n}{not\PYZus{}found\PYZus{}woman}\PY{p}{[}\PY{n}{not\PYZus{}found\PYZus{}woman} \PY{o}{==} \PY{l+s+s2}{\PYZdq{}}\PY{l+s+s2}{Cuerpos no encontrados}\PY{l+s+s2}{\PYZdq{}}\PY{p}{]}
\PY{n+nb}{print}\PY{p}{(}\PY{l+s+sa}{f}\PY{l+s+s2}{\PYZdq{}}\PY{l+s+s2}{Mujeres que desaparecieron entre las edades de 15 a 35 años:  }\PY{l+s+si}{\PYZob{}}\PY{n+nb}{len}\PY{p}{(}\PY{n}{not\PYZus{}found\PYZus{}woman}\PY{p}{)}\PY{l+s+si}{\PYZcb{}}\PY{l+s+s2}{\PYZdq{}}\PY{p}{)}
\end{Verbatim}
\end{tcolorbox}

    \begin{Verbatim}[commandchars=\\\{\}]
Mujeres que desaparecieron entre las edades de 15 a 35 años:  51
    \end{Verbatim}

    \begin{enumerate}
\def\labelenumi{\roman{enumi})}
\setcounter{enumi}{3}
\tightlist
\item
  ¿Cuantos hombres mayores a 20 años sobrevivieron?
\end{enumerate}

    \begin{tcolorbox}[breakable, size=fbox, boxrule=1pt, pad at break*=1mm,colback=cellbackground, colframe=cellborder]
\prompt{In}{incolor}{398}{\boxspacing}
\begin{Verbatim}[commandchars=\\\{\}]
\PY{n}{men\PYZus{}20} \PY{o}{=} \PY{n}{data}\PY{p}{[}\PY{p}{(}\PY{n}{data}\PY{o}{.}\PY{n}{age} \PY{o}{\PYZgt{}} \PY{l+m+mi}{20}\PY{p}{)} \PY{o}{\PYZam{}} \PY{p}{(}\PY{n}{data}\PY{o}{.}\PY{n}{survived} \PY{o}{==} \PY{l+m+mi}{1}\PY{p}{)}\PY{p}{]}
\PY{n+nb}{print}\PY{p}{(}\PY{l+s+sa}{f}\PY{l+s+s2}{\PYZdq{}}\PY{l+s+s2}{Total de hombres mayores de 20 que sobrevivieron: }\PY{l+s+si}{\PYZob{}}\PY{n+nb}{len}\PY{p}{(}\PY{n}{men\PYZus{}20}\PY{p}{)}\PY{l+s+si}{\PYZcb{}}\PY{l+s+s2}{\PYZdq{}}\PY{p}{)}
\end{Verbatim}
\end{tcolorbox}

    \begin{Verbatim}[commandchars=\\\{\}]
Total de hombres mayores de 20 que sobrevivieron: 313
    \end{Verbatim}

    \begin{tcolorbox}[breakable, size=fbox, boxrule=1pt, pad at break*=1mm,colback=cellbackground, colframe=cellborder]
\prompt{In}{incolor}{399}{\boxspacing}
\begin{Verbatim}[commandchars=\\\{\}]
\PY{n}{sns}\PY{o}{.}\PY{n}{displot}\PY{p}{(}\PY{n}{men\PYZus{}20}\PY{p}{,} \PY{n}{x}\PY{o}{=}\PY{l+s+s2}{\PYZdq{}}\PY{l+s+s2}{age}\PY{l+s+s2}{\PYZdq{}}\PY{p}{,} \PY{n}{y}\PY{o}{=}\PY{l+s+s2}{\PYZdq{}}\PY{l+s+s2}{pclass}\PY{l+s+s2}{\PYZdq{}}\PY{p}{)}
\end{Verbatim}
\end{tcolorbox}

            \begin{tcolorbox}[breakable, size=fbox, boxrule=.5pt, pad at break*=1mm, opacityfill=0]
\prompt{Out}{outcolor}{399}{\boxspacing}
\begin{Verbatim}[commandchars=\\\{\}]
<seaborn.axisgrid.FacetGrid at 0x7f5a90cbe530>
\end{Verbatim}
\end{tcolorbox}
        
    \begin{center}
    \adjustimage{max size={0.9\linewidth}{0.9\paperheight}}{output_17_1.png}
    \end{center}
    { \hspace*{\fill} \\}
    
    \begin{enumerate}
\def\labelenumi{\alph{enumi})}
\setcounter{enumi}{21}
\tightlist
\item
  ¿Cuantas mujeres menores a 25 años sobrevivieron?.
\end{enumerate}

    \begin{tcolorbox}[breakable, size=fbox, boxrule=1pt, pad at break*=1mm,colback=cellbackground, colframe=cellborder]
\prompt{In}{incolor}{400}{\boxspacing}
\begin{Verbatim}[commandchars=\\\{\}]
\PY{n}{woman\PYZus{}25} \PY{o}{=} \PY{n}{data}\PY{p}{[}\PY{p}{(}\PY{n}{data}\PY{o}{.}\PY{n}{age} \PY{o}{\PYZgt{}} \PY{l+m+mi}{25}\PY{p}{)} \PY{o}{\PYZam{}} \PY{p}{(}\PY{n}{data}\PY{o}{.}\PY{n}{survived} \PY{o}{==} \PY{l+m+mi}{1}\PY{p}{)}\PY{p}{]}
\PY{n+nb}{print}\PY{p}{(}\PY{l+s+sa}{f}\PY{l+s+s2}{\PYZdq{}}\PY{l+s+s2}{Total de mujeres mayores de 25 que sobrevivieron: }\PY{l+s+si}{\PYZob{}}\PY{n+nb}{len}\PY{p}{(}\PY{n}{men\PYZus{}20}\PY{p}{)}\PY{l+s+si}{\PYZcb{}}\PY{l+s+s2}{\PYZdq{}}\PY{p}{)}
\end{Verbatim}
\end{tcolorbox}

    \begin{Verbatim}[commandchars=\\\{\}]
Total de mujeres mayores de 25 que sobrevivieron: 313
    \end{Verbatim}

    \begin{tcolorbox}[breakable, size=fbox, boxrule=1pt, pad at break*=1mm,colback=cellbackground, colframe=cellborder]
\prompt{In}{incolor}{401}{\boxspacing}
\begin{Verbatim}[commandchars=\\\{\}]
\PY{n}{sns}\PY{o}{.}\PY{n}{displot}\PY{p}{(}\PY{n}{woman\PYZus{}25}\PY{p}{,} \PY{n}{x}\PY{o}{=}\PY{l+s+s2}{\PYZdq{}}\PY{l+s+s2}{age}\PY{l+s+s2}{\PYZdq{}}\PY{p}{)}
\end{Verbatim}
\end{tcolorbox}

            \begin{tcolorbox}[breakable, size=fbox, boxrule=.5pt, pad at break*=1mm, opacityfill=0]
\prompt{Out}{outcolor}{401}{\boxspacing}
\begin{Verbatim}[commandchars=\\\{\}]
<seaborn.axisgrid.FacetGrid at 0x7f5a8ff1c880>
\end{Verbatim}
\end{tcolorbox}
        
    \begin{center}
    \adjustimage{max size={0.9\linewidth}{0.9\paperheight}}{output_20_1.png}
    \end{center}
    { \hspace*{\fill} \\}
    
    Además, Genere una copia del conjunto de datos y rellene los datos
faltantes (NA's) con un valor de 0 en el caso de datos numéricos usados
como identificador la palabra ``desconocido'' en el caso de datos tipo
cadena de caracteres y en el caso de variables numéricas use el promedio
de los valores de esa columna (p.ej., la edad y la tarifa)

    \begin{tcolorbox}[breakable, size=fbox, boxrule=1pt, pad at break*=1mm,colback=cellbackground, colframe=cellborder]
\prompt{In}{incolor}{402}{\boxspacing}
\begin{Verbatim}[commandchars=\\\{\}]
\PY{c+c1}{\PYZsh{} Generamos  una copia del conjunto de datos originales}
\PY{n}{data\PYZus{}cp} \PY{o}{=} \PY{n}{data}\PY{o}{.}\PY{n}{copy}\PY{p}{(}\PY{p}{)}
\PY{c+c1}{\PYZsh{} Ver el número de elementos faltantes}
\PY{n+nb}{print}\PY{p}{(}\PY{n}{data\PYZus{}cp}\PY{o}{.}\PY{n}{isnull}\PY{p}{(}\PY{p}{)}\PY{o}{.}\PY{n}{sum}\PY{p}{(}\PY{p}{)}\PY{p}{)}
\end{Verbatim}
\end{tcolorbox}

    \begin{Verbatim}[commandchars=\\\{\}]
pclass          0
survived        0
name            0
sex             0
age           263
sibsp           0
parch           0
ticket          0
fare            1
cabin        1014
embarked        2
boat          823
body         1188
home.dest     564
dtype: int64
    \end{Verbatim}

    \begin{tcolorbox}[breakable, size=fbox, boxrule=1pt, pad at break*=1mm,colback=cellbackground, colframe=cellborder]
\prompt{In}{incolor}{403}{\boxspacing}
\begin{Verbatim}[commandchars=\\\{\}]
\PY{k}{def} \PY{n+nf}{fill\PYZus{}NA}\PY{p}{(}\PY{n}{data}\PY{p}{,} \PY{n}{columns\PYZus{}names}\PY{p}{,} \PY{n}{values}\PY{p}{)}\PY{p}{:}
    \PY{l+s+sd}{\PYZdq{}\PYZdq{}\PYZdq{}Rellena los NA del dataframe original\PYZdq{}\PYZdq{}\PYZdq{}}
    \PY{k}{for} \PY{n}{column\PYZus{}name}\PY{p}{,} \PY{n}{value} \PY{o+ow}{in} \PY{n+nb}{zip}\PY{p}{(}\PY{n}{columns\PYZus{}names}\PY{p}{,} \PY{n}{values}\PY{p}{)}\PY{p}{:}
        \PY{c+c1}{\PYZsh{} Por paso de referencia actualizaremos el dataframe}
        \PY{n}{data}\PY{p}{[}\PY{n}{column\PYZus{}name}\PY{p}{]} \PY{o}{=} \PY{n}{data}\PY{p}{[}\PY{n}{column\PYZus{}name}\PY{p}{]}\PY{o}{.}\PY{n}{fillna}\PY{p}{(}\PY{n}{value}\PY{p}{)}
\end{Verbatim}
\end{tcolorbox}

    \begin{tcolorbox}[breakable, size=fbox, boxrule=1pt, pad at break*=1mm,colback=cellbackground, colframe=cellborder]
\prompt{In}{incolor}{404}{\boxspacing}
\begin{Verbatim}[commandchars=\\\{\}]
\PY{c+c1}{\PYZsh{} Para variables numericas 0}
\PY{n}{columns\PYZus{}names} \PY{o}{=} \PY{p}{[}\PY{l+s+s2}{\PYZdq{}}\PY{l+s+s2}{body}\PY{l+s+s2}{\PYZdq{}}\PY{p}{,} \PY{l+s+s2}{\PYZdq{}}\PY{l+s+s2}{boat}\PY{l+s+s2}{\PYZdq{}}\PY{p}{]}
\PY{n}{fill\PYZus{}NA}\PY{p}{(}\PY{n}{data\PYZus{}cp}\PY{p}{,} \PY{n}{colums\PYZus{}names}\PY{p}{,} \PY{p}{[}\PY{l+m+mi}{0}\PY{p}{,}\PY{l+m+mi}{0}\PY{p}{]}\PY{p}{)}
\PY{n}{data\PYZus{}cp}\PY{p}{[}\PY{n}{columns\PYZus{}names}\PY{p}{]}\PY{o}{.}\PY{n}{head}\PY{p}{(}\PY{l+m+mi}{5}\PY{p}{)}
\PY{c+c1}{\PYZsh{} “desconocido\PYZdq{} en el caso de datos tipo cadena de caracteres}
\PY{n}{columns\PYZus{}names} \PY{o}{=} \PY{p}{[}\PY{l+s+s1}{\PYZsq{}}\PY{l+s+s1}{home.dest}\PY{l+s+s1}{\PYZsq{}}\PY{p}{,} \PY{l+s+s1}{\PYZsq{}}\PY{l+s+s1}{embarked}\PY{l+s+s1}{\PYZsq{}}\PY{p}{,} \PY{l+s+s1}{\PYZsq{}}\PY{l+s+s1}{cabin}\PY{l+s+s1}{\PYZsq{}}\PY{p}{]}
\PY{n}{fill\PYZus{}NA}\PY{p}{(}\PY{n}{data\PYZus{}cp}\PY{p}{,} \PY{n}{columns\PYZus{}names}\PY{p}{,} \PY{p}{[}\PY{l+s+s2}{\PYZdq{}}\PY{l+s+s2}{desconocido}\PY{l+s+s2}{\PYZdq{}}\PY{p}{]}\PY{o}{*}\PY{l+m+mi}{3}\PY{p}{)}
\PY{c+c1}{\PYZsh{} promedio de los valores para age y fare}
\PY{n}{columns\PYZus{}names} \PY{o}{=} \PY{p}{[}\PY{l+s+s1}{\PYZsq{}}\PY{l+s+s1}{age}\PY{l+s+s1}{\PYZsq{}}\PY{p}{,} \PY{l+s+s1}{\PYZsq{}}\PY{l+s+s1}{fare}\PY{l+s+s1}{\PYZsq{}}\PY{p}{]}
\PY{n}{age\PYZus{}mean} \PY{o}{=} \PY{n}{data}\PY{o}{.}\PY{n}{age}\PY{o}{.}\PY{n}{mean}\PY{p}{(}\PY{p}{)}
\PY{n}{fare\PYZus{}mean} \PY{o}{=} \PY{n}{data}\PY{o}{.}\PY{n}{fare}\PY{o}{.}\PY{n}{mean}\PY{p}{(}\PY{p}{)}
\PY{n}{fill\PYZus{}NA}\PY{p}{(}\PY{n}{data\PYZus{}cp}\PY{p}{,} \PY{n}{columns\PYZus{}names}\PY{p}{,} \PY{p}{[}\PY{n}{age\PYZus{}mean}\PY{p}{,} \PY{n}{fare\PYZus{}mean}\PY{p}{]}\PY{p}{)}
\end{Verbatim}
\end{tcolorbox}

    \begin{tcolorbox}[breakable, size=fbox, boxrule=1pt, pad at break*=1mm,colback=cellbackground, colframe=cellborder]
\prompt{In}{incolor}{405}{\boxspacing}
\begin{Verbatim}[commandchars=\\\{\}]
\PY{c+c1}{\PYZsh{} Vemos nuevamente el número de elementos faltantes}
\PY{n+nb}{print}\PY{p}{(}\PY{n}{data\PYZus{}cp}\PY{o}{.}\PY{n}{isnull}\PY{p}{(}\PY{p}{)}\PY{o}{.}\PY{n}{sum}\PY{p}{(}\PY{p}{)}\PY{p}{)}
\end{Verbatim}
\end{tcolorbox}

    \begin{Verbatim}[commandchars=\\\{\}]
pclass       0
survived     0
name         0
sex          0
age          0
sibsp        0
parch        0
ticket       0
fare         0
cabin        0
embarked     0
boat         0
body         0
home.dest    0
dtype: int64
    \end{Verbatim}

    \begin{tcolorbox}[breakable, size=fbox, boxrule=1pt, pad at break*=1mm,colback=cellbackground, colframe=cellborder]
\prompt{In}{incolor}{406}{\boxspacing}
\begin{Verbatim}[commandchars=\\\{\}]
\PY{n}{data\PYZus{}cp}\PY{o}{.}\PY{n}{tail}\PY{p}{(}\PY{p}{)}
\end{Verbatim}
\end{tcolorbox}

            \begin{tcolorbox}[breakable, size=fbox, boxrule=.5pt, pad at break*=1mm, opacityfill=0]
\prompt{Out}{outcolor}{406}{\boxspacing}
\begin{Verbatim}[commandchars=\\\{\}]
      pclass  survived                       name     sex        age  sibsp  \textbackslash{}
1304       3         0       Zabour, Miss. Hileni  female  14.500000      1
1305       3         0      Zabour, Miss. Thamine  female  29.881135      1
1306       3         0  Zakarian, Mr. Mapriededer    male  26.500000      0
1307       3         0        Zakarian, Mr. Ortin    male  27.000000      0
1308       3         0         Zimmerman, Mr. Leo    male  29.000000      0

      parch  ticket     fare        cabin embarked boat   body    home.dest
1304      0    2665  14.4542  desconocido        C    0  328.0  desconocido
1305      0    2665  14.4542  desconocido        C    0    0.0  desconocido
1306      0    2656   7.2250  desconocido        C    0  304.0  desconocido
1307      0    2670   7.2250  desconocido        C    0    0.0  desconocido
1308      0  315082   7.8750  desconocido        S    0    0.0  desconocido
\end{Verbatim}
\end{tcolorbox}
        
    \hypertarget{normalizaciuxf3n-y-estandarizaciuxf3n}{%
\subsection{Normalización y
Estandarización}\label{normalizaciuxf3n-y-estandarizaciuxf3n}}

    Finalmente, de los campos ``age'' y ``fare'' agregue columnas al
conjunto de datos qué contengan los valores normalizados.

    \begin{tcolorbox}[breakable, size=fbox, boxrule=1pt, pad at break*=1mm,colback=cellbackground, colframe=cellborder]
\prompt{In}{incolor}{407}{\boxspacing}
\begin{Verbatim}[commandchars=\\\{\}]
\PY{n}{data\PYZus{}cp}\PY{p}{[}\PY{l+s+s1}{\PYZsq{}}\PY{l+s+s1}{age\PYZus{}norm}\PY{l+s+s1}{\PYZsq{}}\PY{p}{]} \PY{o}{=} \PY{p}{(}\PY{n}{data\PYZus{}cp}\PY{p}{[}\PY{l+s+s1}{\PYZsq{}}\PY{l+s+s1}{age}\PY{l+s+s1}{\PYZsq{}}\PY{p}{]} \PY{o}{\PYZhy{}} \PY{n}{data}\PY{o}{.}\PY{n}{age}\PY{o}{.}\PY{n}{mean}\PY{p}{(}\PY{p}{)} \PY{p}{)} \PY{o}{/} \PY{p}{(}\PY{n}{data}\PY{o}{.}\PY{n}{age}\PY{o}{.}\PY{n}{std}\PY{p}{(}\PY{p}{)}\PY{p}{)}

\PY{n}{data\PYZus{}cp}\PY{p}{[}\PY{l+s+s1}{\PYZsq{}}\PY{l+s+s1}{fare\PYZus{}norm2}\PY{l+s+s1}{\PYZsq{}}\PY{p}{]} \PY{o}{=} \PY{p}{(}\PY{n}{data\PYZus{}cp}\PY{p}{[}\PY{l+s+s1}{\PYZsq{}}\PY{l+s+s1}{fare}\PY{l+s+s1}{\PYZsq{}}\PY{p}{]} \PY{o}{\PYZhy{}} \PY{n}{np}\PY{o}{.}\PY{n}{min}\PY{p}{(}\PY{n}{data}\PY{o}{.}\PY{n}{fare}\PY{p}{)}\PY{p}{)} \PY{o}{/} \PY{p}{(}\PY{n}{np}\PY{o}{.}\PY{n}{max}\PY{p}{(}\PY{n}{data}\PY{o}{.}\PY{n}{fare}\PY{p}{)} \PY{o}{\PYZhy{}} \PY{n}{np}\PY{o}{.}\PY{n}{min}\PY{p}{(}\PY{n}{data}\PY{o}{.}\PY{n}{fare}\PY{p}{)}\PY{p}{)}

\PY{n}{data\PYZus{}cp}\PY{o}{.}\PY{n}{tail}\PY{p}{(}\PY{l+m+mi}{5}\PY{p}{)}
\end{Verbatim}
\end{tcolorbox}

            \begin{tcolorbox}[breakable, size=fbox, boxrule=.5pt, pad at break*=1mm, opacityfill=0]
\prompt{Out}{outcolor}{407}{\boxspacing}
\begin{Verbatim}[commandchars=\\\{\}]
      pclass  survived                       name     sex        age  sibsp  \textbackslash{}
1304       3         0       Zabour, Miss. Hileni  female  14.500000      1
1305       3         0      Zabour, Miss. Thamine  female  29.881135      1
1306       3         0  Zakarian, Mr. Mapriededer    male  26.500000      0
1307       3         0        Zakarian, Mr. Ortin    male  27.000000      0
1308       3         0         Zimmerman, Mr. Leo    male  29.000000      0

      parch  ticket     fare        cabin embarked boat   body    home.dest  \textbackslash{}
1304      0    2665  14.4542  desconocido        C    0  328.0  desconocido
1305      0    2665  14.4542  desconocido        C    0    0.0  desconocido
1306      0    2656   7.2250  desconocido        C    0  304.0  desconocido
1307      0    2670   7.2250  desconocido        C    0    0.0  desconocido
1308      0  315082   7.8750  desconocido        S    0    0.0  desconocido

      age\_norm  fare\_norm2
1304 -1.067134    0.028213
1305  0.000000    0.028213
1306 -0.234581    0.014102
1307 -0.199891    0.014102
1308 -0.061133    0.015371
\end{Verbatim}
\end{tcolorbox}
        
    \begin{enumerate}
\def\labelenumi{\alph{enumi})}
\setcounter{enumi}{1}
\tightlist
\item
  Utilizando el archivo ``movies.csv'' construya una o varias funciones
  qué permitan calcular una matriz de distancias para los datos
  numéricos en el dataFrame. La función debe permitir construir la
  matriz de distancia usando las distancias de Manhattan, Euclideana y
  de Minkowski (para p igual a 3).
\end{enumerate}

    \begin{tcolorbox}[breakable, size=fbox, boxrule=1pt, pad at break*=1mm,colback=cellbackground, colframe=cellborder]
\prompt{In}{incolor}{408}{\boxspacing}
\begin{Verbatim}[commandchars=\\\{\}]
\PY{k+kn}{import} \PY{n+nn}{plotly} \PY{k}{as} \PY{n+nn}{py}
\PY{k+kn}{import} \PY{n+nn}{sympy} \PY{k}{as} \PY{n+nn}{sp}
\PY{k+kn}{import} \PY{n+nn}{plotly}\PY{n+nn}{.}\PY{n+nn}{graph\PYZus{}objs} \PY{k}{as} \PY{n+nn}{go}
\PY{k+kn}{from} \PY{n+nn}{plotly}\PY{n+nn}{.}\PY{n+nn}{offline} \PY{k+kn}{import} \PY{n}{init\PYZus{}notebook\PYZus{}mode}\PY{p}{,} \PY{n}{iplot}
\PY{k+kn}{from} \PY{n+nn}{scipy}\PY{n+nn}{.}\PY{n+nn}{spatial} \PY{k+kn}{import} \PY{n}{distance\PYZus{}matrix}
\end{Verbatim}
\end{tcolorbox}

    \begin{tcolorbox}[breakable, size=fbox, boxrule=1pt, pad at break*=1mm,colback=cellbackground, colframe=cellborder]
\prompt{In}{incolor}{409}{\boxspacing}
\begin{Verbatim}[commandchars=\\\{\}]
\PY{n}{movies\PYZus{}df} \PY{o}{=} \PY{n}{pd}\PY{o}{.}\PY{n}{read\PYZus{}csv}\PY{p}{(}\PY{l+s+s2}{\PYZdq{}}\PY{l+s+s2}{data/movies.csv}\PY{l+s+s2}{\PYZdq{}}\PY{p}{,} \PY{n}{sep}\PY{o}{=}\PY{l+s+s2}{\PYZdq{}}\PY{l+s+s2}{;}\PY{l+s+s2}{\PYZdq{}}\PY{p}{)}
\PY{n}{movies\PYZus{}df}
\end{Verbatim}
\end{tcolorbox}

            \begin{tcolorbox}[breakable, size=fbox, boxrule=.5pt, pad at break*=1mm, opacityfill=0]
\prompt{Out}{outcolor}{409}{\boxspacing}
\begin{Verbatim}[commandchars=\\\{\}]
   user\_id  star\_wars  lord\_of\_the\_rings  harry\_potter
0        1        1.2                4.9           2.1
1        2        2.1                8.1           7.9
2        3        7.4                3.0           9.9
3        4        5.6                0.5           1.8
4        5        1.5                8.3           2.6
5        6        2.5                3.7           6.5
6        7        2.0                8.2           8.5
7        8        1.8                9.3           4.5
8        9        2.6                1.7           3.1
9       10        1.5                4.7           2.3
\end{Verbatim}
\end{tcolorbox}
        
    \begin{tcolorbox}[breakable, size=fbox, boxrule=1pt, pad at break*=1mm,colback=cellbackground, colframe=cellborder]
\prompt{In}{incolor}{410}{\boxspacing}
\begin{Verbatim}[commandchars=\\\{\}]
\PY{c+c1}{\PYZsh{} Iris\PYZhy{}setosa}
\PY{n}{movies\PYZus{}3d} \PY{o}{=} \PY{n}{go}\PY{o}{.}\PY{n}{Scatter3d}\PY{p}{(}
                        \PY{n}{x} \PY{o}{=} \PY{n}{movies\PYZus{}df}\PY{p}{[}\PY{l+s+s1}{\PYZsq{}}\PY{l+s+s1}{star\PYZus{}wars}\PY{l+s+s1}{\PYZsq{}}\PY{p}{]}\PY{p}{,}
                        \PY{n}{y} \PY{o}{=} \PY{n}{movies\PYZus{}df}\PY{p}{[}\PY{l+s+s1}{\PYZsq{}}\PY{l+s+s1}{lord\PYZus{}of\PYZus{}the\PYZus{}rings}\PY{l+s+s1}{\PYZsq{}}\PY{p}{]}\PY{p}{,}
                        \PY{n}{z} \PY{o}{=} \PY{n}{movies\PYZus{}df}\PY{p}{[}\PY{l+s+s1}{\PYZsq{}}\PY{l+s+s1}{harry\PYZus{}potter}\PY{l+s+s1}{\PYZsq{}}\PY{p}{]}\PY{p}{,}
                        \PY{n}{mode} \PY{o}{=} \PY{l+s+s1}{\PYZsq{}}\PY{l+s+s1}{markers}\PY{l+s+s1}{\PYZsq{}}\PY{p}{,}
                        \PY{n}{opacity} \PY{o}{=} \PY{l+m+mf}{0.7}\PY{p}{,}
                        \PY{n}{name} \PY{o}{=} \PY{l+s+s2}{\PYZdq{}}\PY{l+s+s2}{Movies}\PY{l+s+s2}{\PYZdq{}}\PY{p}{,}
                        \PY{n}{marker} \PY{o}{=} \PY{n+nb}{dict}\PY{p}{(}
                                    \PY{n}{size} \PY{o}{=} \PY{l+m+mi}{5}\PY{p}{,}
                                    \PY{n}{color} \PY{o}{=} \PY{l+s+s1}{\PYZsq{}}\PY{l+s+s1}{rgba(255,102, 255,0.8)}\PY{l+s+s1}{\PYZsq{}}\PY{p}{)}
\PY{p}{)}
\PY{n}{movies\PYZus{}3d} \PY{o}{=} \PY{p}{[}\PY{n}{movies\PYZus{}3d}\PY{p}{]}
\PY{n}{fig\PYZus{}3d} \PY{o}{=} \PY{n}{go}\PY{o}{.}\PY{n}{Figure}\PY{p}{(}\PY{n}{data} \PY{o}{=} \PY{n}{movies\PYZus{}3d}\PY{p}{)}
\PY{n}{iplot}\PY{p}{(}\PY{n}{movies\PYZus{}3d}\PY{p}{)}
\end{Verbatim}
\end{tcolorbox}

    \begin{center}
    \adjustimage{max size={0.9\linewidth}{0.9\paperheight}}{output_33_0.png}
    \end{center}
    { \hspace*{\fill} \\}
    
    Ya que nuestro conjunto de datos no tiene una adecuada descripción del
significado de los valores de las columnas, podemos inferir que cada
valor es el grado de satisfaccion de cada usuario respecto a cada
pelicual. Viendo nuestro grafico en 3D podemos ver hay usuarios con la
misma opinion viendo las distancias entre los puntos, para ser más
precicos sacaremos la matriz de distancias.

    \begin{tcolorbox}[breakable, size=fbox, boxrule=1pt, pad at break*=1mm,colback=cellbackground, colframe=cellborder]
\prompt{In}{incolor}{411}{\boxspacing}
\begin{Verbatim}[commandchars=\\\{\}]
\PY{k}{def} \PY{n+nf}{distancia\PYZus{}euclidiana}\PY{p}{(}\PY{n}{a}\PY{p}{,} \PY{n}{b}\PY{p}{)}\PY{p}{:}
    \PY{l+s+sd}{\PYZdq{}\PYZdq{}\PYZdq{}Calcula la distancia euclidiana,}
\PY{l+s+sd}{    }
\PY{l+s+sd}{    Arguments}
\PY{l+s+sd}{    p \PYZhy{} una tupla que será el primer punto.}
\PY{l+s+sd}{    q \PYZhy{} una tupla que será el segundo punto.}
\PY{l+s+sd}{    }
\PY{l+s+sd}{    Return    }
\PY{l+s+sd}{    Regresa un entero con la distancia entre los puntos.}
\PY{l+s+sd}{    \PYZdq{}\PYZdq{}\PYZdq{}} 
    \PY{c+c1}{\PYZsh{} Convertimos los puntos en vectores de numpy}
    \PY{n}{p1} \PY{o}{=} \PY{n}{np}\PY{o}{.}\PY{n}{array}\PY{p}{(}\PY{n}{a}\PY{p}{)}
    \PY{n}{p2} \PY{o}{=} \PY{n}{np}\PY{o}{.}\PY{n}{array}\PY{p}{(}\PY{n}{b}\PY{p}{)}
    \PY{k}{return} \PY{n}{np}\PY{o}{.}\PY{n}{sqrt}\PY{p}{(}\PY{n}{np}\PY{o}{.}\PY{n}{sum}\PY{p}{(}\PY{p}{(}\PY{n}{p1}\PY{o}{\PYZhy{}}\PY{n}{p2}\PY{p}{)}\PY{o}{*}\PY{o}{*}\PY{l+m+mi}{2}\PY{p}{)}\PY{p}{)}    

\PY{k}{def} \PY{n+nf}{distancia\PYZus{}manhattan}\PY{p}{(}\PY{n}{a}\PY{p}{,} \PY{n}{b}\PY{p}{)}\PY{p}{:}
    \PY{l+s+sd}{\PYZdq{}\PYZdq{}\PYZdq{}Calcula la distancia manhattan,}
\PY{l+s+sd}{    }
\PY{l+s+sd}{    Arguments}
\PY{l+s+sd}{    p \PYZhy{} una tupla que será el primer punto.}
\PY{l+s+sd}{    q \PYZhy{} una tupla que será el segundo punto.}
\PY{l+s+sd}{    }
\PY{l+s+sd}{    Return    }
\PY{l+s+sd}{    Regresa un entero con la distancia entre los puntos.}
\PY{l+s+sd}{    \PYZdq{}\PYZdq{}\PYZdq{}} 
    \PY{c+c1}{\PYZsh{} Convertimos los puntos en vectores de numpy}
    \PY{n}{p1} \PY{o}{=} \PY{n}{np}\PY{o}{.}\PY{n}{array}\PY{p}{(}\PY{n}{a}\PY{p}{)}
    \PY{n}{p2} \PY{o}{=} \PY{n}{np}\PY{o}{.}\PY{n}{array}\PY{p}{(}\PY{n}{b}\PY{p}{)}
    \PY{k}{return} \PY{n}{np}\PY{o}{.}\PY{n}{sum}\PY{p}{(}\PY{n}{np}\PY{o}{.}\PY{n}{abs}\PY{p}{(}\PY{n}{p1}\PY{o}{\PYZhy{}}\PY{n}{p2}\PY{p}{)}\PY{p}{)}

\PY{k}{def} \PY{n+nf}{distancia\PYZus{}minkowski}\PY{p}{(}\PY{n}{a}\PY{p}{,} \PY{n}{b}\PY{p}{,} \PY{n}{p}\PY{o}{=}\PY{l+m+mi}{3}\PY{p}{)}\PY{p}{:}
    \PY{l+s+sd}{\PYZdq{}\PYZdq{}\PYZdq{}Calcula la distancia minkowski,}
\PY{l+s+sd}{    }
\PY{l+s+sd}{    Arguments}
\PY{l+s+sd}{    p \PYZhy{} una tupla que será el primer punto.}
\PY{l+s+sd}{    q \PYZhy{} una tupla que será el segundo punto.}
\PY{l+s+sd}{    }
\PY{l+s+sd}{    Return    }
\PY{l+s+sd}{    Regresa un entero con la distancia entre los puntos.}
\PY{l+s+sd}{    \PYZdq{}\PYZdq{}\PYZdq{}} 
    \PY{n}{p1} \PY{o}{=} \PY{n}{np}\PY{o}{.}\PY{n}{array}\PY{p}{(}\PY{n}{a}\PY{p}{)}
    \PY{n}{p2} \PY{o}{=} \PY{n}{np}\PY{o}{.}\PY{n}{array}\PY{p}{(}\PY{n}{b}\PY{p}{)}
    \PY{k}{return} \PY{n}{np}\PY{o}{.}\PY{n}{sum}\PY{p}{(}\PY{p}{(}\PY{n}{np}\PY{o}{.}\PY{n}{abs}\PY{p}{(}\PY{n}{p1}\PY{o}{\PYZhy{}}\PY{n}{p2}\PY{p}{)}\PY{o}{*}\PY{o}{*}\PY{n}{p}\PY{p}{)}\PY{p}{)} \PY{o}{*}\PY{o}{*} \PY{p}{(}\PY{l+m+mi}{1}\PY{o}{/}\PY{n}{p}\PY{p}{)}
\end{Verbatim}
\end{tcolorbox}

    \begin{tcolorbox}[breakable, size=fbox, boxrule=1pt, pad at break*=1mm,colback=cellbackground, colframe=cellborder]
\prompt{In}{incolor}{412}{\boxspacing}
\begin{Verbatim}[commandchars=\\\{\}]
\PY{k}{def} \PY{n+nf}{obtener\PYZus{}matriz\PYZus{}distancias}\PY{p}{(}\PY{n}{data}\PY{p}{,} \PY{n}{func\PYZus{}dist}\PY{p}{)}\PY{p}{:}
    \PY{l+s+sd}{\PYZdq{}\PYZdq{}\PYZdq{}Crea una matriz de distancia\PYZdq{}\PYZdq{}\PYZdq{}}
    \PY{n}{matriz\PYZus{}distancias} \PY{o}{=} \PY{n}{np}\PY{o}{.}\PY{n}{zeros}\PY{p}{(}\PY{p}{(}\PY{n+nb}{len}\PY{p}{(}\PY{n}{data}\PY{p}{)}\PY{p}{,} \PY{n+nb}{len}\PY{p}{(}\PY{n}{data}\PY{p}{)}\PY{p}{)}\PY{p}{)}
    \PY{k}{for} \PY{n}{i}\PY{p}{,} \PY{n}{usuario} \PY{o+ow}{in} \PY{n+nb}{enumerate}\PY{p}{(}\PY{n}{data}\PY{p}{)}\PY{p}{:}
        \PY{n}{x1}\PY{p}{,} \PY{n}{y1}\PY{p}{,} \PY{n}{z1} \PY{o}{=} \PY{n}{usuario}
        \PY{k}{for} \PY{n}{j}\PY{p}{,} \PY{n}{usuario2} \PY{o+ow}{in} \PY{n+nb}{enumerate}\PY{p}{(}\PY{n}{data}\PY{p}{)}\PY{p}{:}
            \PY{c+c1}{\PYZsh{} Aprovechando las propiedades de simetria e identidad de los indiscernibles}
            \PY{k}{if} \PY{n}{j} \PY{o}{\PYZgt{}} \PY{n}{i}\PY{p}{:}
                \PY{n}{x2}\PY{p}{,} \PY{n}{y2}\PY{p}{,} \PY{n}{z2} \PY{o}{=} \PY{n}{usuario2}
                \PY{n}{distancia} \PY{o}{=} \PY{n}{func\PYZus{}dist}\PY{p}{(}\PY{p}{(}\PY{n}{x1}\PY{p}{,} \PY{n}{y1}\PY{p}{,} \PY{n}{z1}\PY{p}{)}\PY{p}{,} \PY{p}{(}\PY{n}{x2}\PY{p}{,} \PY{n}{y2}\PY{p}{,} \PY{n}{z2}\PY{p}{)}\PY{p}{)}                        
                \PY{n}{matriz\PYZus{}distancias}\PY{p}{[}\PY{n}{i}\PY{p}{]}\PY{p}{[}\PY{n}{j}\PY{p}{]} \PY{o}{=} \PY{n}{distancia}
                \PY{n}{matriz\PYZus{}distancias}\PY{p}{[}\PY{n}{j}\PY{p}{]}\PY{p}{[}\PY{n}{i}\PY{p}{]} \PY{o}{=} \PY{n}{distancia}
            
    \PY{k}{return} \PY{n}{matriz\PYZus{}distancias}    
\end{Verbatim}
\end{tcolorbox}

    \hypertarget{distancia-euclidiana}{%
\subsection{Distancia Euclidiana}\label{distancia-euclidiana}}

    \begin{tcolorbox}[breakable, size=fbox, boxrule=1pt, pad at break*=1mm,colback=cellbackground, colframe=cellborder]
\prompt{In}{incolor}{413}{\boxspacing}
\begin{Verbatim}[commandchars=\\\{\}]
\PY{n}{data} \PY{o}{=} \PY{n}{movies\PYZus{}df}\PY{o}{.}\PY{n}{iloc}\PY{p}{[}\PY{p}{:}\PY{p}{,}\PY{l+m+mi}{1}\PY{p}{:}\PY{p}{]}\PY{o}{.}\PY{n}{values}
\PY{n}{matriz\PYZus{}dist} \PY{o}{=} \PY{n}{obtener\PYZus{}matriz\PYZus{}distancias}\PY{p}{(}\PY{n}{data}\PY{p}{,} \PY{n}{distancia\PYZus{}euclidiana}\PY{p}{)}
\PY{n+nb}{print}\PY{p}{(}\PY{l+s+s2}{\PYZdq{}}\PY{l+s+se}{\PYZbs{}t}\PY{l+s+se}{\PYZbs{}t}\PY{l+s+se}{\PYZbs{}t}\PY{l+s+s2}{ Matriz de distancia de euclidines}\PY{l+s+s2}{\PYZdq{}}\PY{p}{)}
\PY{n}{display}\PY{p}{(}\PY{n}{sp}\PY{o}{.}\PY{n}{Matrix}\PY{p}{(}\PY{n}{matriz\PYZus{}dist}\PY{p}{)}\PY{p}{)}
\end{Verbatim}
\end{tcolorbox}

    \begin{Verbatim}[commandchars=\\\{\}]
                         Matriz de distancia de euclidines
    \end{Verbatim}

    $\displaystyle \left[\begin{matrix}0 & 6.68505796534331 & 10.1434708063858 & 6.22976725086901 & 3.44963766213207 & 4.74236228055175 & 7.24499827467198 & 5.04777178564959 & 3.63318042491699 & 0.412310562561766\\6.68505796534331 & 0 & 7.6223356000638 & 10.3547090736534 & 5.33760245803301 & 4.63465209050259 & 0.616441400296897 & 3.61801050302511 & 8.0156097709407 & 6.57875368135941\\10.1434708063858 & 7.6223356000638 & 0 & 8.66602561731732 & 10.779146533933 & 6.00499791840097 & 7.62627038597505 & 10.0104944932805 & 8.42436941260294 & 9.77036335045939\\6.22976725086901 & 10.3547090736534 & 8.66602561731732 & 0 & 8.84816365128946 & 6.47610994347687 & 10.8231233939192 & 9.95841352826845 & 3.4828149534536 & 5.89067059000926\\3.44963766213207 & 5.33760245803301 & 10.779146533933 & 8.84816365128946 & 0 & 6.11310068623117 & 5.92199290779717 & 2.16794833886788 & 6.7096944788865 & 3.61247837363769\\4.74236228055175 & 4.63465209050259 & 6.00499791840097 & 6.47610994347687 & 6.11310068623117 & 0 & 4.94974746830583 & 5.98748695196908 & 3.94588393138977 & 4.43170396123207\\7.24499827467198 & 0.616441400296897 & 7.62627038597505 & 10.8231233939192 & 5.92199290779717 & 4.94974746830583 & 0 & 4.15331193145904 & 8.4717176534632 & 7.13722635202219\\5.04777178564959 & 3.61801050302511 & 10.0104944932805 & 9.95841352826845 & 2.16794833886788 & 5.98748695196908 & 4.15331193145904 & 0 & 7.76916983982201 & 5.10783711564885\\3.63318042491699 & 8.0156097709407 & 8.42436941260294 & 3.4828149534536 & 6.7096944788865 & 3.94588393138977 & 8.4717176534632 & 7.76916983982201 & 0 & 3.29393381840012\\0.412310562561766 & 6.57875368135941 & 9.77036335045939 & 5.89067059000926 & 3.61247837363769 & 4.43170396123207 & 7.13722635202219 & 5.10783711564885 & 3.29393381840012 & 0\end{matrix}\right]$

    
    Para comprobar los resultados utilizaremos
scipy.spatial.distance\_matrix
\textasciitilde\textasciitilde\textasciitilde{}

p = 1, Manhattan Distance p = 2, Euclidean Distance p = ∞, Chebychev
Distance

\textasciitilde\textasciitilde\textasciitilde{}

    \begin{tcolorbox}[breakable, size=fbox, boxrule=1pt, pad at break*=1mm,colback=cellbackground, colframe=cellborder]
\prompt{In}{incolor}{414}{\boxspacing}
\begin{Verbatim}[commandchars=\\\{\}]
\PY{c+c1}{\PYZsh{} Comprobacion con scipy.spatial}
\PY{n}{dist\PYZus{}mat\PYZus{}scpy} \PY{o}{=} \PY{n}{distance\PYZus{}matrix}\PY{p}{(}\PY{n}{data}\PY{p}{,} \PY{n}{data}\PY{p}{,} \PY{n}{p}\PY{o}{=}\PY{l+m+mi}{2}\PY{p}{)}
\PY{n+nb}{print}\PY{p}{(}\PY{l+s+s2}{\PYZdq{}}\PY{l+s+se}{\PYZbs{}t}\PY{l+s+se}{\PYZbs{}t}\PY{l+s+se}{\PYZbs{}t}\PY{l+s+s2}{ Matriz de distancia de euclidines con Scipy}\PY{l+s+s2}{\PYZdq{}}\PY{p}{)}
\PY{n}{display}\PY{p}{(}\PY{n}{sp}\PY{o}{.}\PY{n}{Matrix}\PY{p}{(}\PY{n}{dist\PYZus{}mat\PYZus{}scpy}\PY{p}{)}\PY{p}{)}
\end{Verbatim}
\end{tcolorbox}

    \begin{Verbatim}[commandchars=\\\{\}]
                         Matriz de distancia de euclidines con Scipy
    \end{Verbatim}

    $\displaystyle \left[\begin{matrix}0 & 6.68505796534331 & 10.1434708063858 & 6.22976725086901 & 3.44963766213207 & 4.74236228055175 & 7.24499827467198 & 5.04777178564959 & 3.63318042491699 & 0.412310562561766\\6.68505796534331 & 0 & 7.6223356000638 & 10.3547090736534 & 5.33760245803301 & 4.63465209050259 & 0.616441400296897 & 3.61801050302511 & 8.0156097709407 & 6.57875368135941\\10.1434708063858 & 7.6223356000638 & 0 & 8.66602561731732 & 10.779146533933 & 6.00499791840097 & 7.62627038597505 & 10.0104944932805 & 8.42436941260294 & 9.77036335045939\\6.22976725086901 & 10.3547090736534 & 8.66602561731732 & 0 & 8.84816365128946 & 6.47610994347687 & 10.8231233939192 & 9.95841352826845 & 3.4828149534536 & 5.89067059000926\\3.44963766213207 & 5.33760245803301 & 10.779146533933 & 8.84816365128946 & 0 & 6.11310068623117 & 5.92199290779717 & 2.16794833886788 & 6.7096944788865 & 3.61247837363769\\4.74236228055175 & 4.63465209050259 & 6.00499791840097 & 6.47610994347687 & 6.11310068623117 & 0 & 4.94974746830583 & 5.98748695196908 & 3.94588393138977 & 4.43170396123207\\7.24499827467198 & 0.616441400296897 & 7.62627038597505 & 10.8231233939192 & 5.92199290779717 & 4.94974746830583 & 0 & 4.15331193145904 & 8.4717176534632 & 7.13722635202219\\5.04777178564959 & 3.61801050302511 & 10.0104944932805 & 9.95841352826845 & 2.16794833886788 & 5.98748695196908 & 4.15331193145904 & 0 & 7.76916983982201 & 5.10783711564885\\3.63318042491699 & 8.0156097709407 & 8.42436941260294 & 3.4828149534536 & 6.7096944788865 & 3.94588393138977 & 8.4717176534632 & 7.76916983982201 & 0 & 3.29393381840012\\0.412310562561766 & 6.57875368135941 & 9.77036335045939 & 5.89067059000926 & 3.61247837363769 & 4.43170396123207 & 7.13722635202219 & 5.10783711564885 & 3.29393381840012 & 0\end{matrix}\right]$

    
    \hypertarget{distancia-manhattan}{%
\subsection{Distancia Manhattan}\label{distancia-manhattan}}

    \begin{tcolorbox}[breakable, size=fbox, boxrule=1pt, pad at break*=1mm,colback=cellbackground, colframe=cellborder]
\prompt{In}{incolor}{415}{\boxspacing}
\begin{Verbatim}[commandchars=\\\{\}]
\PY{n}{matriz\PYZus{}dist} \PY{o}{=} \PY{n}{obtener\PYZus{}matriz\PYZus{}distancias}\PY{p}{(}\PY{n}{data}\PY{p}{,} \PY{n}{distancia\PYZus{}manhattan}\PY{p}{)}
\PY{n+nb}{print}\PY{p}{(}\PY{l+s+s2}{\PYZdq{}}\PY{l+s+se}{\PYZbs{}t}\PY{l+s+se}{\PYZbs{}t}\PY{l+s+se}{\PYZbs{}t}\PY{l+s+s2}{ Matriz de distancia de manhattan}\PY{l+s+s2}{\PYZdq{}}\PY{p}{)}
\PY{n}{display}\PY{p}{(}\PY{n}{sp}\PY{o}{.}\PY{n}{Matrix}\PY{p}{(}\PY{n}{matriz\PYZus{}dist}\PY{p}{)}\PY{p}{)}
\end{Verbatim}
\end{tcolorbox}

    \begin{Verbatim}[commandchars=\\\{\}]
                         Matriz de distancia de manhattan
    \end{Verbatim}

    $\displaystyle \left[\begin{matrix}0 & 9.9 & 15.9 & 9.1 & 4.2 & 6.9 & 10.5 & 7.4 & 5.6 & 0.7\\9.9 & 0 & 12.4 & 17.2 & 6.1 & 6.2 & 0.799999999999999 & 4.9 & 11.7 & 9.6\\15.9 & 12.4 & 0 & 12.4 & 18.5 & 9.0 & 12.0 & 17.3 & 12.9 & 15.2\\9.1 & 17.2 & 12.4 & 0 & 12.7 & 11.0 & 18.0 & 15.3 & 5.5 & 8.8\\4.2 & 6.1 & 18.5 & 12.7 & 0 & 9.5 & 6.5 & 3.2 & 8.2 & 3.9\\6.9 & 6.2 & 9.0 & 11.0 & 9.5 & 0 & 7.0 & 8.3 & 5.5 & 6.2\\10.5 & 0.799999999999999 & 12.0 & 18.0 & 6.5 & 7.0 & 0 & 5.3 & 12.5 & 10.2\\7.4 & 4.9 & 17.3 & 15.3 & 3.2 & 8.3 & 5.3 & 0 & 9.8 & 7.1\\5.6 & 11.7 & 12.9 & 5.5 & 8.2 & 5.5 & 12.5 & 9.8 & 0 & 4.9\\0.7 & 9.6 & 15.2 & 8.8 & 3.9 & 6.2 & 10.2 & 7.1 & 4.9 & 0\end{matrix}\right]$

    
    \begin{tcolorbox}[breakable, size=fbox, boxrule=1pt, pad at break*=1mm,colback=cellbackground, colframe=cellborder]
\prompt{In}{incolor}{416}{\boxspacing}
\begin{Verbatim}[commandchars=\\\{\}]
\PY{n}{dist\PYZus{}mat\PYZus{}scpy} \PY{o}{=} \PY{n}{distance\PYZus{}matrix}\PY{p}{(}\PY{n}{data}\PY{p}{,} \PY{n}{data}\PY{p}{,} \PY{n}{p}\PY{o}{=}\PY{l+m+mi}{1}\PY{p}{)}
\PY{n+nb}{print}\PY{p}{(}\PY{l+s+s2}{\PYZdq{}}\PY{l+s+se}{\PYZbs{}t}\PY{l+s+se}{\PYZbs{}t}\PY{l+s+se}{\PYZbs{}t}\PY{l+s+s2}{ Matriz de distancia de manhattan con Scipy}\PY{l+s+s2}{\PYZdq{}}\PY{p}{)}
\PY{n}{display}\PY{p}{(}\PY{n}{sp}\PY{o}{.}\PY{n}{Matrix}\PY{p}{(}\PY{n}{dist\PYZus{}mat\PYZus{}scpy}\PY{p}{)}\PY{p}{)}
\end{Verbatim}
\end{tcolorbox}

    \begin{Verbatim}[commandchars=\\\{\}]
                         Matriz de distancia de manhattan con Scipy
    \end{Verbatim}

    $\displaystyle \left[\begin{matrix}0 & 9.9 & 15.9 & 9.1 & 4.2 & 6.9 & 10.5 & 7.4 & 5.6 & 0.7\\9.9 & 0 & 12.4 & 17.2 & 6.1 & 6.2 & 0.799999999999999 & 4.9 & 11.7 & 9.6\\15.9 & 12.4 & 0 & 12.4 & 18.5 & 9.0 & 12.0 & 17.3 & 12.9 & 15.2\\9.1 & 17.2 & 12.4 & 0 & 12.7 & 11.0 & 18.0 & 15.3 & 5.5 & 8.8\\4.2 & 6.1 & 18.5 & 12.7 & 0 & 9.5 & 6.5 & 3.2 & 8.2 & 3.9\\6.9 & 6.2 & 9.0 & 11.0 & 9.5 & 0 & 7.0 & 8.3 & 5.5 & 6.2\\10.5 & 0.799999999999999 & 12.0 & 18.0 & 6.5 & 7.0 & 0 & 5.3 & 12.5 & 10.2\\7.4 & 4.9 & 17.3 & 15.3 & 3.2 & 8.3 & 5.3 & 0 & 9.8 & 7.1\\5.6 & 11.7 & 12.9 & 5.5 & 8.2 & 5.5 & 12.5 & 9.8 & 0 & 4.9\\0.7 & 9.6 & 15.2 & 8.8 & 3.9 & 6.2 & 10.2 & 7.1 & 4.9 & 0\end{matrix}\right]$

    
    \hypertarget{distancia-minkowski}{%
\subsection{Distancia Minkowski}\label{distancia-minkowski}}

    \begin{tcolorbox}[breakable, size=fbox, boxrule=1pt, pad at break*=1mm,colback=cellbackground, colframe=cellborder]
\prompt{In}{incolor}{417}{\boxspacing}
\begin{Verbatim}[commandchars=\\\{\}]
\PY{n}{matriz\PYZus{}dist} \PY{o}{=} \PY{n}{obtener\PYZus{}matriz\PYZus{}distancias}\PY{p}{(}\PY{n}{data}\PY{p}{,} \PY{n}{distancia\PYZus{}minkowski}\PY{p}{)}
\PY{n+nb}{print}\PY{p}{(}\PY{l+s+s2}{\PYZdq{}}\PY{l+s+se}{\PYZbs{}t}\PY{l+s+se}{\PYZbs{}t}\PY{l+s+se}{\PYZbs{}t}\PY{l+s+s2}{ Matriz de distancia de minkowski}\PY{l+s+s2}{\PYZdq{}}\PY{p}{)}
\PY{n}{display}\PY{p}{(}\PY{n}{sp}\PY{o}{.}\PY{n}{Matrix}\PY{p}{(}\PY{n}{matriz\PYZus{}dist}\PY{p}{)}\PY{p}{)}
\end{Verbatim}
\end{tcolorbox}

    \begin{Verbatim}[commandchars=\\\{\}]
                         Matriz de distancia de minkowski
    \end{Verbatim}

    $\displaystyle \left[\begin{matrix}0 & 6.11454916428861 & 8.96172635607764 & 5.5439454576181 & 3.40437729172903 & 4.46656703819761 & 6.68384762858978 & 4.62955139976453 & 3.31750720120221 & 0.350339806038672\\6.11454916428861 & 0 & 6.61551294283105 & 8.91622676181095 & 5.30265679067893 & 4.44782539038182 & 0.601846165480645 & 3.44987047052723 & 7.19751071354172 & 5.99198930956623\\8.96172635607764 & 6.61551294283105 & 0 & 8.20757776606394 & 9.05835740907213 & 5.39807887765857 & 6.70039353150005 & 8.35451128926497 & 7.53154711786155 & 8.65913810366208\\5.5439454576181 & 8.91622676181095 & 8.20757776606394 & 0 & 8.1632351727204 & 5.50007713390447 & 9.29843886386646 & 9.10987536394208 & 3.13884523510554 & 5.23095442295267\\3.40437729172903 & 5.30265679067893 & 9.05835740907213 & 8.1632351727204 & 0 & 5.40218247400413 & 5.90120630318665 & 1.99045451433078 & 6.61112296419712 & 3.60069431052831\\4.46656703819761 & 4.44782539038182 & 5.39807887765857 & 5.50007713390447 & 5.40218247400413 & 0 & 4.62995573011611 & 5.68731146982036 & 3.61661558064796 & 4.23745783293259\\6.68384762858978 & 0.601846165480645 & 6.70039353150005 & 9.29843886386646 & 5.90120630318665 & 4.62995573011611 & 0 & 4.02770351913122 & 7.56130493603761 & 6.55245911199444\\4.62955139976453 & 3.44987047052723 & 8.35451128926497 & 9.10987536394208 & 1.99045451433078 & 5.68731146982036 & 4.02770351913122 & 0 & 7.61874412977636 & 4.76236483015256\\3.31750720120221 & 7.19751071354172 & 7.53154711786155 & 3.13884523510554 & 6.61112296419712 & 3.61661558064796 & 7.56130493603761 & 7.61874412977636 & 0 & 3.06676249438003\\0.350339806038672 & 5.99198930956623 & 8.65913810366208 & 5.23095442295267 & 3.60069431052831 & 4.23745783293259 & 6.55245911199444 & 4.76236483015256 & 3.06676249438003 & 0\end{matrix}\right]$

    
    \begin{tcolorbox}[breakable, size=fbox, boxrule=1pt, pad at break*=1mm,colback=cellbackground, colframe=cellborder]
\prompt{In}{incolor}{418}{\boxspacing}
\begin{Verbatim}[commandchars=\\\{\}]
\PY{n}{dist\PYZus{}mat\PYZus{}scpy} \PY{o}{=} \PY{n}{distance\PYZus{}matrix}\PY{p}{(}\PY{n}{data}\PY{p}{,} \PY{n}{data}\PY{p}{,} \PY{n}{p}\PY{o}{=}\PY{l+m+mi}{3}\PY{p}{)}
\PY{n+nb}{print}\PY{p}{(}\PY{l+s+s2}{\PYZdq{}}\PY{l+s+se}{\PYZbs{}t}\PY{l+s+se}{\PYZbs{}t}\PY{l+s+se}{\PYZbs{}t}\PY{l+s+s2}{ Matriz de distancia de minkowski con Scipy}\PY{l+s+s2}{\PYZdq{}}\PY{p}{)}
\PY{n}{display}\PY{p}{(}\PY{n}{sp}\PY{o}{.}\PY{n}{Matrix}\PY{p}{(}\PY{n}{dist\PYZus{}mat\PYZus{}scpy}\PY{p}{)}\PY{p}{)}
\end{Verbatim}
\end{tcolorbox}

    \begin{Verbatim}[commandchars=\\\{\}]
                         Matriz de distancia de minkowski con Scipy
    \end{Verbatim}

    $\displaystyle \left[\begin{matrix}0 & 6.11454916428861 & 8.96172635607764 & 5.5439454576181 & 3.40437729172903 & 4.46656703819761 & 6.68384762858978 & 4.62955139976453 & 3.31750720120221 & 0.350339806038672\\6.11454916428861 & 0 & 6.61551294283105 & 8.91622676181095 & 5.30265679067893 & 4.44782539038182 & 0.601846165480645 & 3.44987047052723 & 7.19751071354172 & 5.99198930956623\\8.96172635607764 & 6.61551294283105 & 0 & 8.20757776606394 & 9.05835740907213 & 5.39807887765857 & 6.70039353150005 & 8.35451128926497 & 7.53154711786155 & 8.65913810366208\\5.5439454576181 & 8.91622676181095 & 8.20757776606394 & 0 & 8.1632351727204 & 5.50007713390447 & 9.29843886386646 & 9.10987536394208 & 3.13884523510554 & 5.23095442295267\\3.40437729172903 & 5.30265679067893 & 9.05835740907213 & 8.1632351727204 & 0 & 5.40218247400413 & 5.90120630318665 & 1.99045451433078 & 6.61112296419712 & 3.60069431052831\\4.46656703819761 & 4.44782539038182 & 5.39807887765857 & 5.50007713390447 & 5.40218247400413 & 0 & 4.62995573011611 & 5.68731146982036 & 3.61661558064796 & 4.23745783293259\\6.68384762858978 & 0.601846165480645 & 6.70039353150005 & 9.29843886386646 & 5.90120630318665 & 4.62995573011611 & 0 & 4.02770351913122 & 7.56130493603761 & 6.55245911199444\\4.62955139976453 & 3.44987047052723 & 8.35451128926497 & 9.10987536394208 & 1.99045451433078 & 5.68731146982036 & 4.02770351913122 & 0 & 7.61874412977636 & 4.76236483015256\\3.31750720120221 & 7.19751071354172 & 7.53154711786155 & 3.13884523510554 & 6.61112296419712 & 3.61661558064796 & 7.56130493603761 & 7.61874412977636 & 0 & 3.06676249438003\\0.350339806038672 & 5.99198930956623 & 8.65913810366208 & 5.23095442295267 & 3.60069431052831 & 4.23745783293259 & 6.55245911199444 & 4.76236483015256 & 3.06676249438003 & 0\end{matrix}\right]$

    
    Además, Usando los métodos ``dendrogram'' y ``linkage'' construya un
diagrama en forma de árbol (dendrograma) para el conjunto de datos en
``movies.csv''. Repita el proceso ahora usando algún esquema de
normalización del rango de los datos.

    \begin{tcolorbox}[breakable, size=fbox, boxrule=1pt, pad at break*=1mm,colback=cellbackground, colframe=cellborder]
\prompt{In}{incolor}{419}{\boxspacing}
\begin{Verbatim}[commandchars=\\\{\}]
\PY{k+kn}{from} \PY{n+nn}{scipy}\PY{n+nn}{.}\PY{n+nn}{cluster}\PY{n+nn}{.}\PY{n+nn}{hierarchy} \PY{k+kn}{import} \PY{n}{dendrogram}\PY{p}{,}\PY{n}{linkage}
\end{Verbatim}
\end{tcolorbox}

    \begin{tcolorbox}[breakable, size=fbox, boxrule=1pt, pad at break*=1mm,colback=cellbackground, colframe=cellborder]
\prompt{In}{incolor}{420}{\boxspacing}
\begin{Verbatim}[commandchars=\\\{\}]
\PY{n}{Z} \PY{o}{=} \PY{n}{linkage}\PY{p}{(}\PY{n}{data}\PY{p}{,} \PY{l+s+s2}{\PYZdq{}}\PY{l+s+s2}{ward}\PY{l+s+s2}{\PYZdq{}}\PY{p}{)}
\PY{n}{plt}\PY{o}{.}\PY{n}{figure}\PY{p}{(}\PY{n}{figsize}\PY{o}{=}\PY{p}{(}\PY{l+m+mi}{25}\PY{p}{,}\PY{l+m+mi}{10}\PY{p}{)}\PY{p}{)}
\PY{n}{plt}\PY{o}{.}\PY{n}{title}\PY{p}{(}\PY{l+s+s2}{\PYZdq{}}\PY{l+s+s2}{Dendrograma del clustering jerárquico}\PY{l+s+s2}{\PYZdq{}}\PY{p}{)}
\PY{n}{plt}\PY{o}{.}\PY{n}{xlabel}\PY{p}{(}\PY{l+s+s2}{\PYZdq{}}\PY{l+s+s2}{Índices de la Muestra}\PY{l+s+s2}{\PYZdq{}}\PY{p}{)}
\PY{n}{plt}\PY{o}{.}\PY{n}{ylabel}\PY{p}{(}\PY{l+s+s2}{\PYZdq{}}\PY{l+s+s2}{Distancias}\PY{l+s+s2}{\PYZdq{}}\PY{p}{)}
\PY{n}{dendrogram}\PY{p}{(}\PY{n}{Z}\PY{p}{,} \PY{n}{leaf\PYZus{}rotation}\PY{o}{=}\PY{l+m+mf}{90.}\PY{p}{,} \PY{n}{leaf\PYZus{}font\PYZus{}size}\PY{o}{=}\PY{l+m+mf}{8.0}\PY{p}{,} \PY{n}{color\PYZus{}threshold}\PY{o}{=}\PY{l+m+mf}{0.7}\PY{o}{*}\PY{l+m+mi}{180}\PY{p}{)}
\PY{n}{plt}\PY{o}{.}\PY{n}{show}\PY{p}{(}\PY{p}{)}
\end{Verbatim}
\end{tcolorbox}

    \begin{center}
    \adjustimage{max size={0.9\linewidth}{0.9\paperheight}}{output_49_0.png}
    \end{center}
    { \hspace*{\fill} \\}
    
    \hypertarget{normalizando-los-datos.}{%
\subsubsection{Normalizando los datos.}\label{normalizando-los-datos.}}

    \begin{tcolorbox}[breakable, size=fbox, boxrule=1pt, pad at break*=1mm,colback=cellbackground, colframe=cellborder]
\prompt{In}{incolor}{421}{\boxspacing}
\begin{Verbatim}[commandchars=\\\{\}]
\PY{n}{data\PYZus{}norm} \PY{o}{=} \PY{n}{np}\PY{o}{.}\PY{n}{zeros}\PY{p}{(}\PY{p}{(}\PY{n}{data}\PY{o}{.}\PY{n}{shape}\PY{p}{[}\PY{l+m+mi}{0}\PY{p}{]}\PY{p}{,} \PY{n}{data}\PY{o}{.}\PY{n}{shape}\PY{p}{[}\PY{l+m+mi}{1}\PY{p}{]}\PY{p}{)}\PY{p}{)}
\PY{k}{for} \PY{n}{col} \PY{o+ow}{in} \PY{n+nb}{range}\PY{p}{(}\PY{n}{data}\PY{o}{.}\PY{n}{shape}\PY{p}{[}\PY{l+m+mi}{1}\PY{p}{]}\PY{p}{)}\PY{p}{:}
    \PY{n}{data\PYZus{}norm}\PY{p}{[}\PY{p}{:}\PY{p}{,}\PY{n}{col}\PY{p}{]} \PY{o}{=} \PY{p}{(}\PY{n}{data}\PY{p}{[}\PY{p}{:}\PY{p}{,}\PY{n}{col}\PY{p}{]} \PY{o}{\PYZhy{}} \PY{n}{data}\PY{p}{[}\PY{p}{:}\PY{p}{,}\PY{n}{col}\PY{p}{]}\PY{o}{.}\PY{n}{mean}\PY{p}{(}\PY{p}{)}\PY{p}{)} \PY{o}{/} \PY{n}{data}\PY{p}{[}\PY{p}{:}\PY{p}{,}\PY{n}{col}\PY{p}{]}\PY{o}{.}\PY{n}{std}\PY{p}{(}\PY{p}{)}
\PY{n}{data\PYZus{}norm}
\end{Verbatim}
\end{tcolorbox}

            \begin{tcolorbox}[breakable, size=fbox, boxrule=.5pt, pad at break*=1mm, opacityfill=0]
\prompt{Out}{outcolor}{421}{\boxspacing}
\begin{Verbatim}[commandchars=\\\{\}]
array([[-0.83997603, -0.11622047, -0.98229819],
       [-0.37332268,  0.97761925,  1.03803142],
       [ 2.37474706, -0.7656878 ,  1.7346968 ],
       [ 1.44144035, -1.62025008, -1.08679799],
       [-0.68442492,  1.04598423, -0.80813184],
       [-0.16592119, -0.52641036,  0.55036565],
       [-0.42517305,  1.01180174,  1.24703103],
       [-0.5288738 ,  1.38780914, -0.14629973],
       [-0.11407082, -1.21006019, -0.6339655 ],
       [-0.68442492, -0.18458545, -0.91263165]])
\end{Verbatim}
\end{tcolorbox}
        
    \begin{tcolorbox}[breakable, size=fbox, boxrule=1pt, pad at break*=1mm,colback=cellbackground, colframe=cellborder]
\prompt{In}{incolor}{422}{\boxspacing}
\begin{Verbatim}[commandchars=\\\{\}]
\PY{n}{Z} \PY{o}{=} \PY{n}{linkage}\PY{p}{(}\PY{n}{data\PYZus{}norm}\PY{p}{,} \PY{l+s+s2}{\PYZdq{}}\PY{l+s+s2}{ward}\PY{l+s+s2}{\PYZdq{}}\PY{p}{)}
\PY{n}{plt}\PY{o}{.}\PY{n}{figure}\PY{p}{(}\PY{n}{figsize}\PY{o}{=}\PY{p}{(}\PY{l+m+mi}{25}\PY{p}{,}\PY{l+m+mi}{10}\PY{p}{)}\PY{p}{)}
\PY{n}{plt}\PY{o}{.}\PY{n}{title}\PY{p}{(}\PY{l+s+s2}{\PYZdq{}}\PY{l+s+s2}{Dendrograma del clustering jerárquico}\PY{l+s+s2}{\PYZdq{}}\PY{p}{)}
\PY{n}{plt}\PY{o}{.}\PY{n}{xlabel}\PY{p}{(}\PY{l+s+s2}{\PYZdq{}}\PY{l+s+s2}{Índices de la Muestra}\PY{l+s+s2}{\PYZdq{}}\PY{p}{)}
\PY{n}{plt}\PY{o}{.}\PY{n}{ylabel}\PY{p}{(}\PY{l+s+s2}{\PYZdq{}}\PY{l+s+s2}{Distancias}\PY{l+s+s2}{\PYZdq{}}\PY{p}{)}
\PY{n}{dendrogram}\PY{p}{(}\PY{n}{Z}\PY{p}{,} \PY{n}{leaf\PYZus{}rotation}\PY{o}{=}\PY{l+m+mf}{90.}\PY{p}{,} \PY{n}{leaf\PYZus{}font\PYZus{}size}\PY{o}{=}\PY{l+m+mf}{8.0}\PY{p}{,} \PY{n}{color\PYZus{}threshold}\PY{o}{=}\PY{l+m+mf}{0.7}\PY{o}{*}\PY{l+m+mi}{180}\PY{p}{)}
\PY{n}{plt}\PY{o}{.}\PY{n}{show}\PY{p}{(}\PY{p}{)}
\end{Verbatim}
\end{tcolorbox}

    \begin{center}
    \adjustimage{max size={0.9\linewidth}{0.9\paperheight}}{output_52_0.png}
    \end{center}
    { \hspace*{\fill} \\}
    
    \begin{enumerate}
\def\labelenumi{\roman{enumi})}
\item
  ¿Que diferencias puede encontrar en los resultados previos?. Podemos
  ver que al normalizar las variables los dendrogramas se apachurran en
  el eje y, es decir, la altura de los rectuangulos se hacen más
  pequeñas, pero sigue conservando el mismo orden en eje x.
\item
  ¿En qué casos resulta importante llevar a cabo un proceso de
  normalización del rango de datos?. Es muy importante cuando el rango
  de los datos es superior a los demas, ya que esto puede sesgar a los
  dato pequeños.
\item
  Consulte los diferentes tipos de distancias que se pueden usar como
  parámetro en el método ``linkage'', ¿qué características de los datos
  se podría basar uno para elegir una determinada distancia?.
\end{enumerate}

Existen otros tipos de enlazamiento, cada uno resuelve soluciones
distintas:

\begin{itemize}
\tightlist
\item
  Single linkage: utiliza la distancia más corta entre dos vectores de
  cada grupo.
\item
  Complete linkage: toma en cuenta la distancia más grande entre dos
  vectores de cada grupo.
\item
  Average linkage: usa la distancia promedio entre todos los vectores de
  cada grupo.
\item
  Ward linkage: la distancia es la suma de las diferencias al cuadrado
  en los grupos.
\end{itemize}

    \hypertarget{referencias}{%
\subsubsection{Referencias}\label{referencias}}

\begin{itemize}
\tightlist
\item
  https://docs.scipy.org/doc/scipy/reference/spatial.distance.html
\item
  https://www.encyclopedia-titanica.org/
\item
  https://www.kaggle.com/vinicius150987/titanic3
\end{itemize}


    % Add a bibliography block to the postdoc
    
    
    
\end{document}
